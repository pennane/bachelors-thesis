% Material and Methods
%\clearpage%if the chapter heading starts close to bottom of the page, force a line break and remove the vertical vspace
\vspace{21.5pt}
\chapter{Kokemukset}

% Vähä kokemusperäsiä juttuja siitä, et miten nelososan jutut kokemusten perusteella toimii.

% On helppo kirjoittaa näennäisesti oikeista tavoista toimia. On huomattavasti vaikeampaa seurata jotain blaa blaa.

% Seuraavaksi pyritään perustelemaan sitä miten blaa blaa blaa.

\section{Luettavuus}

Yli 20 vuotta JavaScriptia ohjelmoinut Archibald kuvaa ohjelmoijan ajattelun muuttumista uran aikana kolmen vaiheen kautta. Aluksi hän kirjoitti yksinkertaista koodia, välttäen monimutkaisempia työkaluja, kuten säännöllisiä lausekkeita (regexp), listametodeja kuten \texttt{reduce} ja rekursiota, sillä ne tuntuivat uhkaavilta. \citep{is_reduce_bad}

Kehittyessään ohjelmoijana hän alkoi aktiivisesti hyödyntää näitä työkaluja. Ajatuksena oli osoittaa kollegoille omaa osaamistaan käyttämällä koodissa monimutkaisempia rakenteita. Tällöin hän uskoi, että koodin tulisi heijastaa taitavuutta ohjelmoijana. \citep{is_reduce_bad}

Nyt hän keskittyy kirjoittamaan koodia, joka muistuttaa ohjelmoinnin perusteiden oppikirjaesimerkkejä: selkeää, yksinkertaista ja helposti ymmärrettävää. \citep{is_reduce_bad}

Hän suosii for-silmukoiden käyttämistä funktionaalisen \texttt{reduce}-funktion sijasta, ja perustelee valintaa luettavuudella \cite{is_reduce_bad}. Luettavuuden kannalta merkittävimpiä tekijöitä ovat keskimääräinen rivin pituus, tunnisteiden määrä per rivi ja keskimääräinen sulkeiden määrä \cite[8]{busereadability}. Dijkstran mukaan ohjelman luettavuus riippuu suurelta osin sen \glsdisp{ohjausrakenne}{ohjausrakenteiden} yksinkertaisuudesta \cite{dijkstra1976discipline}. Ohjelmoijat oppivat ensin for-silmukoita, ennen kuin ymmärtävät, mitä \texttt{reduce} (tai \texttt{fold}) tarkoittaa. Kuitenkin tunnisteet ja rakenteet, jotka ohjelmoija tuntee, ovat henkilökohtaisia ja perustuvat kokemuksiin sekä opiskelutaustaan.

Voi siis ajatella, että luettavuus on tarpeellista suhteuttaa työympäristöön. Jos koodikannassa suositaan funktioiden yhdistelemistä, tätä käytäntöä kannattaa noudattaa. Jos taas for-silmukoita käytetään mieluummin \texttt{reduce}-funktion sijasta, ne ovat parempi valinta.

\section{Uudelleenkäytettävyys}

% DRY malli. hyvät ja pahat !
% Tee kolmesti, sitten vasta abstraktio.
% löydettävyys big thing

\section{Ylläpidettävyys}


\section{Tehokkuus}
% Ellei blaa blaa koodata c-koodia tai pelimoottoreita tai kuka ties mitä, niin tehokkuutta ei parane miettiä liikaa. Varsinkin systeemeissä jotka käyttävät aikaa kaikkialla muualla kuin jossain yhdessä algoritmissa. Pullonkaulana on paljon muita asioita.

% Tiedostaminen on kuitenkin tärkeää. Jos tietää millaiset asiat vaikuttavat tehokkuuteen, voi jo alitajunnallisesti taltuttaa. Tiedon turha kopiointi voi yksinkertaisessa react appissakin aiheuttaa paljon pahaa.

\section{Paradigmaerot ja niiden ristiriidat}