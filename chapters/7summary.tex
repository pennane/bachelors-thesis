\clearpage%if the chapter heading starts close to bottom of the page, force a line break and remove the vertical vspace
\vspace{21.5pt}
\chapter{Yhteenveto}

Psykologi Kurt Lewin kiteytti aikoinaan, että mikään ei ole niin käytännöllistä kuin hyvä teoria. Tämä näkyy funktionaalisessa ohjelmoinnissa sekä hyvässä että pahassa: sen käsitteillä voidaan rakentaa teoreettisesti tehokkaita, luotettavia ja kompakteja ohjelmia, mutta sokea teorian seuraaminen voi johtaa tehokkuushaittoihin, luettavuusongelmiin ja liian nopeaan abstrahointiin.

Ohjelmien mallinnuksessa teoreettisuus ja funktionaalisen ohjelmoinnin deklaratiivisuus ovat hyödyksi. Mallit, jotka eivät valehtele vähentävät ylläpidettävän koodin määrää ja testaamisen tarvetta. Mallintamista voidaan viedä pidemmälle rakenteilla kuten monadeilla, jolloin mallin ja toteutuksen raja hämärtyy.

Työn alkuvaiheilla olin vakuuttunut funktionaalisen ohjelmoinnin merkittävästä
edusta. Nyt koen, että funktionaalinen ohjelmointi on enemmän kuin epäsuora voitto, eikä sitä tule pakottaa jokaiseen tilanteeseen. Kuvainnollisesti sanottuna kaikukammioon seinä on räjäytetty.

Ohjelmointia tehdään yhteistyössä, joten yhteisiä käytäntöjä on syytä noudattaa. Funktionaalisen ohjelmoinnin käsitteet ja käytänteet ovat ilmaisuvoimaisia ja ennustettavia, mutta myös erittäin teoreettisia ja tiukkoja, mikä vaatii opiskelua ja perustelua.

Löydettiin käytänteitä, joita on luonteva ottaa käyttöön tuleviin projekteihin, ja käytänteitä, joiden käyttöä kannattaa harkita tarkasti. Tutkimuksen lähteissä funktionaalisen ohjelmoinnin hyödyt ilmenivät parhaiten alunperin funktionaalisissa ohjelmointikielissä. Kunkin kielen ominaisuudet vaikuttavat merkittävästi siihen, kuinka tehokkaasti ja sujuvasti funktionaalisia periaatteita voidaan soveltaa.