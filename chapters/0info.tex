
% Global information (title of your thesis, your name, degree programme, major, etc.)
\def\bilingual{yes}
\def\thesislang{finnish}
\def\secondlang{english}
\author{Arttu Pennanen}
%License
\def\thesiscopy{by}

%Finnish section, for title/abstract
\def\otsikko{Pragmaattisen funktionaalisen \\ohjelmoinnin arviointi}
\def\tutkinto{Insinööri (AMK)}
\def\kohjelma{Tieto- ja viestintätekniikka}
\def\suuntautumis{Ohjelmistotuotanto}
\def\thesisfi{Insinöörityö}
\def\ohjaajat{
    FM Simo Silander
}
\def\tiivistelma{
    Funktionaalista ohjelmointia tukee vahva teoreettinen matematiikan tausta, ja se eroaa olennaisesti muista ohjelmointiparadigmoista. Tässä insinöörityössä tarkastellaan, miten funktionaalisen ohjelmoinnin periaatteita voidaan pragmaattisesti integroida ohjelmistoprojekteihin.\newline

    Käydään läpi, miten ohjelmistoympäristöön voidaan tuoda funktionaalisen ohjelmoinnin kieliagnostisia piirteitä, jotka eivät ole sidoksissa käytettävään ohjelmointikieleen. Koodiesimerkit, näkemykset ja perustelut pohjautuvat kokemukseen ohjelmistoyrityksessä sekä julkisiin lähteisiin, joissa käsitellään funktionaalisen ohjelmoinnin, kategoriateorian, ongelmien mallintamisen ja ohjelmoinnin periaatteita.\newline

    Tuloksena perustellaan funktionaalisen ohjelmoinnin käyttöä ohjelmien oikeellisuuden ja ylläpidettävyyden parantamiseksi. Näytetään, miten ohjelmat voidaan estää valehtelemasta ja miten operaatioita voidaan ketjuttaa funktioiden ja monadien avulla. Funktionaalista ohjelmointia voidaan liittää muihin ohjelmointiparadigmoihin, kunhan huolehditaan ohjelmiston monimutkaisuuden hallinnasta. Ohjelmoijat yleensä olettavat ohjelmoivansa sitä ohjelmointikieltä, jolla ohjelmoivat.
}
\def\avainsanat{funktionaalinen ohjelmointi, JavaScript, TypeScript, koodin luettavuus, koodin ylläpidettävyys, koodin suorituskyky, kehittäjäkokemus}
\def\aihe{Insinöörityössä tarkastellaan funktionaalisten periaatteiden soveltamista JavaScriptissä perustuen käytännön kokemukseen ja julkisiin lähteisiin. Työssä korostuu ohjelmien oikeellisuuden ja ylläpidettävyyden parantaminen funktioiden ja monadien avulla.}%for the pdf metadata/properties. If not used, empty it and also the \def\subject.

%English section, for title/abstract
\title{Evaluating the Pragmatic Application of Functional Programming}
\def\metropoliadegree{Bachelor of Engineering} % change to your needs, e.g. "master", etc.
\def\metropoliadegreeprogramme{Information and Communication Technology}
\def\metropoliaspecialisation{Software Engineering}
\def\thesisen{Bachelor’s Thesis} % change to your need, e.g. master's
\def\metropoliainstructors{
    Simo Silander, M.Sc.
}
\def\abstract{
    Functional programming is underpinned by a robust theoretical foundation in mathematics, distinguishing it significantly from other programming paradigms. This thesis explores how the principles of functional programming can be pragmatically applied into software projects.\newline

    Center of discussion is introducing language-agnostic features of functional programming into the software environment. Code examples, insights, and justifications are drawn from experiences in a software company, and from public sources focusing on the principles of functional programming, category theory, problem modeling, and programming in general.\newline

    As a result, an argument is drawn in favor of functional programming to enhance the correctness and maintainability of programs. It is demonstrated how programs can be prevented from lying, and how operations can be chained using functions and monads. Functional programming can, and should, be integrated into other programming paradigms, provided that software complexity is taken care of. Programmers typically assume they are coding in the language they are using.
}
\def\metropoliakeywords{functional programming, JavaScript, TypeScript, code readability, code maintainability, code performance, developer experience}
\def\subject{This thesis explores the application of functional principles in JavaScript based on practical experience and public sources. It emphasizes improving the correctness and maintainability of programs through the use of functions and monads.}%for the pdf metadata/properties. If not used, empty it and also the \def\aihe.
