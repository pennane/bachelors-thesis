% Generate the glossary
\makeglossaries

% Acronyms, abbreviations, etc.


% \newacronym{fp}{fp}{Functionaalinen ohjelmointi}

% Glossary entries


\newglossaryentry{ts}{
	name={TypeScript},
	description={ohjelmointikieli (ts). Muuten kuin \gls{js}, mutta staattisella tyypityksellä}
}

\newglossaryentry{js}{
	name={JavaScript},
	description={ohjelmointikieli (js)}
}

\newglossaryentry{functional_programming}{
	name={Funktionaalinen ohjelmointi},
	text={funktionaalinen ohjelmointi},
	description={Ohjelmointiparadigma, joka korostaa laskennan mallintamista funktioiden avulla}
}

\newglossaryentry{programming_paradigm}{
	name={Ohjelmointiparadigma},
	description={Ohjelmoinnin lähestymistapa tai malli, joka määrittelee, miten ohjelmat rakennetaan ja miten ohjelmointiongelmia ratkaistaan. Esimerkkejä tunnetuista ohjelmointiparadigmoista ovat funktionaalinen, olio- ja imperatiivinen ohjelmointi}
}

\newglossaryentry{imperative_programming}{
	name={Imperatiivinen ohjelmointi},
	description={Ohjelmointiparadigma, jossa ohjelmoija antaa tarkat ohjeet siitä, kuinka tehtävä suoritetaan vaihe vaiheelta. Tämä lähestymistapa keskittyy ohjelman tilan hallintaan ja muuttamiseen käskyjen avulla, kuten muuttujien asettaminen ja silmukoiden käyttö}
}

\newglossaryentry{declarative_programming}{
	name={Deklaratiivinen ohjelmointi},
	text={deklaratiivinen ohjelmointi},
	description={Ohjelmointiparadigma, jossa ohjelmoija määrittelee, mitä lopputuloksen tulisi olla, mutta ei yksityiskohtaisesti sitä, miten tämä tulos saavutetaan}
}

\newglossaryentry{object_oriented_programming}{
	name={Olio-ohjelmointi},
	text={olio-ohjelmointi},
	description={Ohjelmointiparadigma, joka perustuu olioiden, eli ohjelmakomponenttien, ympärille, jotka yhdistävät sekä dataa että siihen liittyviä toimintoja. Olio-ohjelmointi korostaa perinnän, kapseloinnin ja polymorfismin kaltaisia periaatteita}
}

\newglossaryentry{set_theory}{
	name={Joukko-oppi},
	text={joukko-oppi},
	description={\protect\engl{set theory} Joukko-oppi on joukkojen ominaisuuksiin perehtynyt matematiikan osa-alue. Joukko-oppi on perustavanlaatuisessa merkityksessä tietorakenteissa. Joukko-opissa olennaisimpia laskutoimituksia ovat esimerkiksi unioni, leikkaus ja joukkoerotus}
}

\newglossaryentry{category_theory}{
	name={Kategoriateoria},
	text={kategoriateoria},
	description={\protect\engl{category theory} Kategoriateoria on matematiikan osa-alue, joka tutkii erittäin yleistävällä tasolla rakenteiden välisiä suhteita ja yhteyksiä. Kategoriateoriaa käytetään funktionaalisen ohjelmoinnin apukeinona}
}

\newglossaryentry{monad}{
	name={Monadi},
	text={monadi},
	plural={Monadit},
	description={\protect\engl{monad} Monadi on abstrakti tietotyyppi, jota käytetään funktionaalisessa ohjelmoinnissa}
}

\newglossaryentry{language_agnostic}{
	name={Kieliagnostinen},
	text={kieliagnostinen},
	description={\protect\engl{language agnostic} Kieliagnostisuudella tarkoitetaan lähestymistapaa, jossa ei keskitytä tiettyyn ohjelmointikieleen, vaan tarkastellaan peruskäsitteitä ja -periaatteita yleisellä tasolla, sovellettavissa eri kieliin riippumatta niiden syntaksista tai erityispiirteistä}
}

\newglossaryentry{correctness}{
	name={Koodin oikeellisuus},

	description={\protect\engl{correctness} Koodin oikeellisuus on ohjelman kyky täyttää sille asetetut vaatimukset virheettömästi. Toisaalta ilman vaatimuksia koodi ei voi olla oikeellista}
}

\newglossaryentry{pure_function}{
	name={Puhdas funktio},
	description={\protect\engl{pure function} Funktiot, jotka eivät vaikuta ohjelman tilaan ja palauttavat aina saman tuloksen samoilla syötteillä}
}

\newglossaryentry{immutable_data}{
	name={Muuttumaton data},
	text={muuttumaton data},
	description={\protect\engl{immutable data} Data, jota ei voi muuttaa sen luomisen jälkeen}
}

\newglossaryentry{higher_order_function}{
name={Korkeamman asteen funktio},
text={korkeamman asteen funktio}
plural={korkeamman asteen funktiot},
description={\protect\engl{higher order function} Funktiot, jotka voivat ottaa toisia funktioita syötteenään tai palauttavat funktioita tuloksena}
}

\newglossaryentry{combinator}{
	name={Kombinaattori},
	description={\protect\engl{combinator} Funktionaalisen ohjelmoinnin rakennuspalikka, joka yhdistää funktioita ilman muuttujien käyttöä},
	plural={Kombinaattorit}
}

\newglossaryentry{composed_function}{
	name={Yhdistetty funktio},
	plural={Yhdistetyt funktiot},
	description={\protect\engl{function composition} Useammasta funktiosta koostettu funktio, jossa funktiot suoritetaan toinen toisensa jälkeen niin, että edellisen paluuarvo annetaan seuraavan syötteeksi}
}