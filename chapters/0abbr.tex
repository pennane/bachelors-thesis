% Generate the glossary
\makeglossaries

% Acronyms, abbreviations, etc.


\newacronym{fp}{fp}{Functionaalinen ohjelmointi}

% Glossary entries


\newglossaryentry{ts}{
	name={TypeScript},
	description={ohjelmointikieli (ts)}
}

\newglossaryentry{js}{
	name={JavaScript},
	description={ohjelmointikieli (js)}
}


\newglossaryentry{functional_programming}{
	name={Funktionaalinen ohjelmointi},
	description={Ohjelmointiparadigma, joka korostaa laskennan mallintamista funktioiden avulla}
}

\newglossaryentry{programming_paradigm}{
	name={Ohjelmointiparadigma},
	description={Ohjelmoinnin lähestymistapa tai malli, joka määrittelee, miten ohjelmat rakennetaan ja miten ohjelmointiongelmia ratkaistaan. Esimerkkejä tunnetuista ohjelmointiparadigmoista ovat funktionaalinen, olio- ja imperatiivinen ohjelmointi}
}

\newglossaryentry{imperative_programming}{
	name={Imperatiivinen ohjelmointi},
	description={Ohjelmointiparadigma, jossa ohjelmoija antaa tarkat ohjeet siitä, kuinka tehtävä suoritetaan vaihe vaiheelta. Tämä lähestymistapa keskittyy ohjelman tilan hallintaan ja muuttamiseen käskyjen avulla, kuten muuttujien asettaminen ja silmukoiden käyttö}
}

\newglossaryentry{declarative_programming}{
	name={Deklaratiivinen ohjelmointi},
	description={Ohjelmointiparadigma, jossa ohjelmoija määrittelee, mitä lopputuloksen tulisi olla, mutta ei yksityiskohtaisesti sitä, miten tämä tulos saavutetaan}
}

\newglossaryentry{object_oriented_programming}{
	name={Olio-ohjelmointi},
	description={Ohjelmointiparadigma, joka perustuu olioiden, eli ohjelmakomponenttien, ympärille, jotka yhdistävät sekä dataa että siihen liittyviä toimintoja. Olio-ohjelmointi korostaa perinnän, kapseloinnin ja polymorfismin kaltaisia periaatteita}
}

\newglossaryentry{set_theory}{
	name={Joukko-oppi},
	description={Joukko-oppi on joukkojen ominaisuuksiin perehtynyt matematiikan osa-alue. Joukko-oppi on perustavanlaatuisessa merkityksessä tietorakenteissa. Joukko-opissa olennaisimpia laskutoimituksia ovat esimerksiksi unioni, leikkaus ja joukkoerotus.}
}

\newglossaryentry{category_theory}{
	name={Kategoriateoria},
	description={Kategoriateoria on matematiikan osa-alue, joka tutkii erittäin yleistävällä tasolla rakenteiden välisiä suhteita ja yhteyksiä. Kategoriateoriaa käytetään funktionaalisen ohjelmoinnin apukeinona.}
}

\newglossaryentry{language_agnostic}{
	name={Kieliagnostinen},
	description={Kieliagnostisuudella tarkoitetaan lähestymistapaa, jossa ei keskitytä tiettyyn ohjelmointikieleen, vaan tarkastellaan peruskäsitteitä ja -periaatteita yleisellä tasolla, sovellettavissa eri kieliin riippumatta niiden syntaksista tai erityispiirteistä.}
}

\newglossaryentry{correctness}{
	name={Koodin oikeellisuus},
	description={Koodin oikeellisuus on ohjelman kyky täyttää sille asetetut vaatimukset virheettömästi. Toisaalta ilman vaatimuksia koodi ei voi olla oikeellista.}
}