% Generate the glossary
\makeglossaries

% Acronyms, abbreviations, etc.

\newacronym{js}{js}{JavaScript}
\newacronym{ts}{ts}{TypeScript}
\newacronym{io}{I/O}{Input/Output}

% Glossary entries

\newglossaryentry{functional_programming}{
	name={funktionaalinen ohjelmointi},
	description={Ohjelmointiparadigma, joka korostaa laskennan mallintamista funktioiden avulla}
}

\newglossaryentry{programming_paradigm}{
	name={ohjelmointiparadigma},
	description={Ohjelmoinnin lähestymistapa tai malli, joka määrittelee, miten ohjelmat rakennetaan ja miten ohjelmointiongelmia ratkaistaan. Esimerkkejä tunnetuista ohjelmointiparadigmoista ovat funktionaalinen, olio- ja imperatiivinen ohjelmointi}
}

\newglossaryentry{imperative_programming}{
	name={imperatiivinen ohjelmointi},
	description={Ohjelmointiparadigma, jossa ohjelmoija antaa tarkat ohjeet siitä, kuinka tehtävä suoritetaan vaihe vaiheelta. Tämä lähestymistapa keskittyy ohjelman tilan hallintaan ja muuttamiseen käskyjen avulla, kuten muuttujien asettaminen ja silmukoiden käyttö}
}

\newglossaryentry{declarative_programming}{
	name={deklaratiivinen ohjelmointi},
	description={Ohjelmointiparadigma, jossa ohjelmoija määrittelee, mitä lopputuloksen tulisi olla, mutta ei yksityiskohtaisesti sitä, miten tämä tulos saavutetaan}
}

\newglossaryentry{object_oriented_programming}{
	name={olio-ohjelmointi},
	description={Ohjelmointiparadigma, joka perustuu olioiden, eli ohjelmakomponenttien, ympärille, jotka yhdistävät sekä dataa että siihen liittyviä toimintoja. Olio-ohjelmointi korostaa perinnän, kapseloinnin ja polymorfismin kaltaisia periaatteita}
}


