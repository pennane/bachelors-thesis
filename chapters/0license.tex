% License of your thesis
% If you wish to explain what it means. When you publish your thesis in https://theseus.fi
% you will be able to choose between some Creative Commons licenses
% https://creativecommons.org
% Adapt this example text to your taste.
% This would also be the right place to explain the license you choose for the code you
% produced for your thesis.

\pagestyle{empty}
\chapter*{Lisenssit}
\ifdefstring{\thesiscopy}{all}{}{%
  \begin{wrapfigure}{r}{0.3\textwidth}
    \vspace{-20pt}
    \doclicenseImage
  \end{wrapfigure}
 }
\IfLanguageName{finnish}{\copyfi}{\copyen}

Tämä tarkoittaa:

\textbf{Voit vapaasti:}
\begin{itemize}
\item Jakaa \textemdash kopioida aineistoa ja levittää sitä edelleen missä tahansa välineessä ja muodossa missä tahansa tarkoituksessa, myös kaupallisesti.
\item Muunnella \textemdash remiksata ja muokata aineistoa sekä luoda sen pohjalta uusia aineistoja missä tahansa tarkoituksessa, myös kaupallisesti.
\end{itemize}

\textbf{Seuraavilla ehdoilla:}
\begin{itemize}
\item Nimeä \textemdash Sinun on mainittava lähde asianmukaisesti, tarjottava linkki lisenssiin sekä merkittävä, mikäli olet tehnyt muutoksia. Voit tehdä yllä olevan millä tahansa kohtuullisella tavalla, mutta et siten, että annat ymmärtää lisenssinantajan suosittelevan sinua tai teoksen käyttöäsi.
\item Ei muita rajoituksia \textemdash Et voi asettaa sellaisia oikeudellisia ehtoja tai teknisiä estoja, jotka estävät oikeudellisesti muita tekemästä mitään sellaista, minkä lisenssi sallii.
\end{itemize}

%Eventually consider few words why you choose such license? E.g. something like:
% I decided to publish my thesis work under the Creative Commons Attribution-ShareAlike 4.0 International License because I strongly believe that you as reader deserve the freedom to copy, share and modify this work and if you do modify it, it is fair to give these same permissions to the others. A copy in electronic form of this work can be found in \url{https://some.place} with the \LaTeX{} source.

\clearpage
