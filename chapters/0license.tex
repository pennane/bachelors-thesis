% License of your thesis
% If you wish to explain what it means. When you publish your thesis in https://theseus.fi
% you will be able to choose between some Creative Commons licenses
% https://creativecommons.org
% Adapt this example text to your taste.
% This would also be the right place to explain the license you choose for the code you
% produced for your thesis.

\pagestyle{empty}
\newgeometry{top=2.04cm,left=4cm}

{\large\textbf{Lisenssit}}

\ifdefstring{\thesiscopy}{all}{}{%
  \begin{wrapfigure}{r}{0.3\textwidth}
    \doclicenseImage
  \end{wrapfigure}
}
\IfLanguageName{finnish}{\copyfi}{\copyen}


\textbf{Voit vapaasti:}
\begin{itemize}
  \item Jakaa {\textemdash} kopioida aineistoa ja levittää sitä edelleen missä tahansa välineessä ja muodossa missä tahansa tarkoituksessa, myös kaupallisesti.
  \item Muunnella {\textemdash} remiksata ja muokata aineistoa sekä luoda sen pohjalta uusia aineistoja missä tahansa tarkoituksessa, myös kaupallisesti.
\end{itemize}

\textbf{Seuraavilla ehdoilla:}
\begin{itemize}
  \item Nimeä {\textemdash} Sinun on mainittava lähde asianmukaisesti, tarjottava linkki lisenssiin sekä merkittävä, mikäli olet tehnyt muutoksia.\\ Voit tehdä yllä olevan millä tahansa kohtuullisella tavalla, mutta et siten, että annat ymmärtää lisenssinantajan suosittelevan sinua tai teoksen käyttöäsi.
  \item Ei muita rajoituksia {\textemdash} Et voi asettaa sellaisia oikeudellisia ehtoja tai teknisiä estoja, jotka estävät oikeudellisesti muita tekemästä mitään sellaista, minkä lisenssi sallii.
\end{itemize}
\bigskip
{\large\textbf{PDF-tiedoston linkit}}

Navigoinnin helpottamiseksi PDF-tiedostossa on käytössä linkkejä:

\begin{itemize}
  \item\fcolorbox{red}{white}{Punaiset ääriviivat}: tiedoston sisäiset linkit sanastoon
  \item\fcolorbox{green}{white}{Vihreät ääriviivat}: tiedoston sisäiset linkit lähdeluetteloon
  \item \fcolorbox{cyan}{white}{Siniset ääriviivat}: ulkoiset linkit verkkosivuille.
\end{itemize}

\noindent Sisäisistä linkeistä voi palata takaisin näppäinyhdistelmällä \key{Alt}+ \key{Vasen nuoli}
(tai \key{Cmd}+ \key{Vasen nuoli}).
\restoregeometry
\clearpage
