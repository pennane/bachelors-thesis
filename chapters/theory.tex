% Theoretical background
%\clearpage % Uncomment if needed to force a page break before the chapter
\vspace{21.5pt}
\chapter{Teoreettinen tausta}

TODO

\section{Funktionaalinen ohjelmointi}

TODO

Nopea historia, lamdakalkyyli. Olennaiset asiat. Immutability ja "pure function"

TODO,

\subsection{Yhdistefunktiot ja niiden merkitys}


TODO

\begin{code}
  \begin{equation}
    f(x) = 2x
  \end{equation}
  \begin{equation}
    g(x) = x + 3
  \end{equation}
  \begin{equation}
    h(x) = f(g(x)) = 2(x + 3) = 2x + 6
  \end{equation}
  \caption{Matemaattinen esimerkki funktiokompositiosta}
  \label{equation:composition}
\end{code}
\bigskip

\begin{code}
  \begin{minted}{haskell}
f :: Int -> Int
f x = 2 * x

g :: Int -> Int
g x = x + 3

h :: Int -> Int
h x = f (g x)
\end{minted}
  \caption{Haskell-esimerkki funktiokompositiosta}
  \label{code:haskell_composition}
\end{code}
\bigskip
\begin{code}
  \begin{minted}{javascript}
function f(x) {
  return 2 * x;
}

function g(x) {
  return x + 3;
}

function h(x) {
  return f(g(x));
}
\end{minted}
  \caption{JavaScript-esimerkki funktiokompositiosta}
  \label{code:javascript_composition}
\end{code}


\subsection{Joukko-oppi}

TODO

Pari idistä siitä, että joukko-oppia pitäisi ajatella aina. Ei vain funktionaalista ohjelmointia harjoittaessa.

\subsection{Kategoriateoria}

TODO

Pari idistä siitä, mitä hyötyy kun lähtee syvään päähänn. Toisaalta myös se, että miten sinne edes pääsee ja miten sen voi erottaa.

\section{Tyypitys}

TODO



\subsection{Parametrinen polymorfismi}

TODO


\subsection{Hindley-Milner tyyppijärjestelmä}

TODO

Feldmanin mukaan, mahdottomat tilat ovat testausta parempaa \cite{impossiblebetter}.

Myös Dijkstran idis abstraktiosta sopii kauniisti tähän, ja esimerkki siitä et mitä \gls{fp} on parhaimmillaan.

\section{Pragmaattisuus}

TODO

Kun funktionaalisen ohjelmoinnin kauneus on matemaattisissa kulmakivissä jotka eivät taivu, miten sitä voi yhdistää nykymaailman pikakoodaukseen? Python, \gls{ts} yms.


\subsection{A monad is a monoid in the category of endofunctors}

TODO

Hyvää bäntteriä siitä, miksei fp ole lähtenyt lentoon, miksi pragmaattisuus on hot. Miten fp näkyy nykymaailmassa ja miten se mielletään. Pohja kategoriateoriassa pelottaa.

Nopea kattaus \textquote{developer advocate} -heppujen mielipiteistä.

\subsubsection{Funktionaalisen ohjelmoinnin ripottelu}

TODO

Idiksiä siitä että juuri otetaan funktionaalisen ohjemoinnin parhaat palat

\section{Funktionaalinen ohjelmointi \& Typescript}


TODO

Koska aijon käyttää Typescriptiä koodiesimerkeissä tms, niin on aiheellista kirjoittaa siitä miten Typescriptissä \gls{fp} näkyy.

\subsection{Ramda.js}

TODO

Koska funktionaalisen ohjelmakoodin tuottaminen \gls{js} tai \gls{ts} ympäristössä vaatii jotain kirjastoa, joko itse ohjelmoitua tai ulkoista, niin tähän on valittu Ramda.js. Olen käyttänyt sitä töissä ja vapaa-ajalla.

\subsection{Rajoitteet}

TODO

Mitkä asiat eivät ole mahdollisia, tai mitkä ovat erittäin hankalia / järjettömiä kun ohjelmoidaan \gls{fp} koodia \gls{ts}-maailmassa.


\section{Vertailu yhteenvetona}

TODO

Ehkä nopea yhteenveto että mikä on täysin \gls{fp} ominaista ja mikä ei. Mahdollisesti jokin kiva taulukko.