% Theoretical background
%\clearpage % Uncomment if needed to force a page break before the chapter
%\vspace{21.5pt} % Consider removing unless extra space is absolutely necessary
\chapter{Theoretical background}

Check Final Year Project Guide for the content of Theoretical background chapter.

Imperatiivinen \ref{code:imperative} vai deklaratiivinen \ref{code:declarative}? Ketä ginee.

Imperatiivisestni: 
\begin{code} 
  \inputminted{javascript}{code/imperative.js} 
  \captionof{listing}{Imperatiivinen tapa \gls{js} koodissa poistaa parilliset ja tuplata parittomat numerot}
  \label{code:imperative}
\end{code}

Sama homma mut "funktionaalisesti":
\begin{code}
  \inputminted{javascript}{code/declarative.js} 
  \captionof{listing}{Funktionaalinen tapa \gls{js} koodissa poistaa parilliset ja tuplata parittomat numerot}
  \label{code:declarative}
\end{code}

Siitä sitten sanomaan. Imperatiivinen \ref{code:imperative} on nopeampi kuin deklaratiivinen \ref{code:declarative}, mut luettavuus sitten toisin päin.