% Theoretical background
%\clearpage % Uncomment if needed to force a page break before the chapter
\vspace{21.5pt} % Consider removing unless extra space is absolutely necessary
\chapter{Teoreettinen tausta}

TODO - pelkkää filleriä ja testausta

da fak on kaan \gls{functional_programming}.

\section{Funktioiden yhdistäminen}

Olkoon \( f: A \rightarrow B \) ja \( g: B \rightarrow C \) kaksi funktiota. Funktioiden \( f \) ja \( g \) yhdistämistä merkitään \( g \circ f \), ja se on funktio joukosta \( A \) joukkoon \( C \), joka määritellään seuraavasti:

\[
  (g \circ f)(x) = g(f(x))
\]

Esimerkiksi, jos \( f(x) = 2x + 3 \) ja \( g(x) = x^2 \), niin yhdistetty funktio \( (g \circ f)(x) \) lasketaan seuraavasti:

\[
  (g \circ f)(x) = g(f(x)) = g(2x + 3) = (2x + 3)^2
\]

Laajennetaan \( (2x + 3)^2 \):

\[
  (g \circ f)(x) = 4x^2 + 12x + 9
\]

\section{Blaa ja blaa}

Imperatiivinen \ref{code:imperative} vai deklaratiivinen \ref{code:declarative}? Ketä ginee.

Imperatiivisestni:

\begin{code}
  \inputminted{javascript}{code/imperative.js}
  \captionof{listing}{Imperatiivinen tapa \gls{js} koodissa poistaa parilliset ja tuplata parittomat numerot}
  \label{code:imperative}
\end{code}

thö declarative:

\begin{code}
  \inputminted{javascript}{code/declarative.js}
  \captionof{listing}{Funktionaalinen tapa \gls{js} koodissa poistaa parilliset ja tuplata parittomat numerot}
  \label{code:declarative}
\end{code}


Siitä sitten sanomaan. Imperatiivinen \ref{code:imperative} on nopeampi kuin deklaratiivinen \ref{code:declarative}, mut luettavuus sitten toisin päin.

\section{Blaa blaa x 2}

Feldmanin mukaan, mahdottomat tilat ovat testausta parempaa \cite{impossiblebetter}.

\subsection{BLAA :D}
ja niin edelleen