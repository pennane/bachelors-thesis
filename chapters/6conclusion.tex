% Material and Methods
%\clearpage%if the chapter heading starts close to bottom of the page, force a line break and remove the vertical vspace
\vspace{21.5pt}
\chapter{Johtopäätökset}
TODO

Älä valehtele koodissa. Minimoi toisto nostamalla null-tarkistukset omaan tietorakenteeseen.
Mallinna oikein ja \glsdisp{correctness}{oikeellisesti}.

Ohjelmointikieli on huomioitava, missä työskennellään. Ei sokeaa sääntöjen noudattamista ilman oikeaa perusteltua syytä. Kaikki käsitteet ovat sellaisia, että ne on opittava. Mitä enemmän käsitteitä käytetään, sitä vaikeampaa uuden ohjelmoijan on saavuttaa syvällinen ymmärrys koodikannasta.

Abstraktioita on hyvä tehdä kerralla kunnolla, eikä pitkin poikin vain fiiliksen mukaan. Esimerkki keskitettynä Result-monadi koodikannassa voi selkeyttää ja vähentää koodin määrää. Toisaalta mitä vähemmän tämänkaltaisia abstraktioita on, todennäköisesti sitä parempi. Jos ohjelmointikieli on oikeasti puhtaasti funktionaalinen, voi erinäköisiä kategoriateorian algebrallisia tietorakenteita käyttää vapaammin. Jos ohjelmointikieli ei tue niitä puhtaasti, niiden käyttämistä kannattaa harkita tarkemmin.

\section{Suositukset}

Lähtökohtaisesti mitä vähemmän koodia, sen parempi. Kollegat vaikuttavat. Pitää haastaa ja olla haastettavana. Oikeellisuus on a \& o. Ei taikaa.


\section{Jatkotutkimus}

- Kategoriateoria.
- fantasyland spec
- Elixir, Haskell, Go
- Rich Hickey \& Richard Feldman
- fp.ts

\section{Loppusanat}

% Miten onnistua kirjoittamaan sitä kuuluisaa funktionaalista koodia sitten loppujen lopuksi?

% Varmaan siten, että yrittää miettiä kategoriateoriaa. Kategoriateoria on kovin teoreettista, mutta teoreettinen = hyvä pohja, jos teoria on oikeasti totta eikä harhaunta.
