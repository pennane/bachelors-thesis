% Material and Methods
%\clearpage%if the chapter heading starts close to bottom of the page, force a line break and remove the vertical vspace
\vspace{21.5pt}
\chapter{Johtopäätökset ja suositukset}

Funktionaalisen ohjelmointi eroaa fundamentaalisesti tavanomaisesti opetetuista ohjelmointiparadigmoista. Tässä luvussa käydään työn keskeiset johtopäätökset ja mahdolliset jatkotutkimuskohteet. Johtopäätöksissä pohditaan funktionaalisen ohjelmoinnin soveltamista perinteisempään ohjelmointiin painottaen maltillisuutta. Jatkon osalta käsitellään mahdollisia aihepiirin syventymisalueita, joilla voi rikastuttaa ohjelmoinnin käsityksen monimuotoisuutta.

\section{Johtopäätökset}

Funktionaalisen ohjelmoinnin periaatteet voidaan integroida perinteisempään ohjelmointiin. Liiallinen pedanttisuus käsitteitä kohtaan kuitenkin saattaa johtaa ohjelman ylläpidettävyyden ongelmiin, erityisesti muuttuvan koodin tuottamisessa asiakaskäyttöön. Ongelmat tulee mallintaa oikein, sillä tämä vähentää tarpeettomien testausvaiheiden ja if-lauseiden tarvetta. Hyvin mallinnettuna ohjelmakoodin määrä vähenee ja ylläpidettävyys paranee.

Akateemiset käsitteet funktionaalisessa ohjelmoinnissa ovat tiukasti määriteltyjä, mutta niiden noudattaminen ei vaadi pilkuntarkkuutta. Sokeaa sääntöjen noudattamista ilman perusteltua syytä tulee välttää. Yhdistetyt funktiot, funktioiden puhtaus, tiedon muuttumattomuus, algebralliset tietotyypit ja rakenteet ovat toimivia työkaluja ohjelmien rakentamisessa. Jos ohjelmointikieli ei tue niitä natiivisti, niiden mukaan ottaminen yleisesti heikentää koodin luettavuutta. On kuitenkin tärkeää kunnioittaa määritettyjä käytänteitä; kokemusten mukaan keskinkertaisetkin käytänteet ovat parempia kuin niiden puuttuminen kokonaan.

On ollut avartavaa huomata, kuinka esimerkiksi \texttt{Promise} ja \texttt{Array} ovat pohjimmiltaan monadirakenteita. Vaikka tämä tieto auttaa ymmärtämään näiden tietorakenteiden samanlaisuuksia, voi olla hyödyllistä olla tuomatta tietoa ilmi jokaisessa koodikatselmuksessa. Tiedon konkreettista hyötyä on vaikea mitata.

\section{Aihepiirin syventäminen}

Jatkuva kehitys on ohjelmointialan koettu vaatimus. Kirjallisuus, ohjelmointikielet, kirjastot, filosofiat ja näiden kaikkien puolesta ja vastaan puhujat eivät tule loppumaan kesken.

Funktionaalinen ohjelmointi on laaja ja monimuotoinen alue, joka kattaa teoreettisia käsitteitä ja käytännön sovelluksia. Alan jatkotutkimus voisi käsittää useita näkökulmia, ohjelmointikieliä ja tunnettujen ohjelmoijien julkaisuja. Moniulotteinen lähestymistapa on oleellinen, sillä pelkkä funktionaalisen ohjelmoinnin periaatteiden tutkiminen ei välttämättä riitä pragmaattisten ohjelmointitaitojen kehittämiseen.

\subsection{Kategoriateoria}

Kategoriateorian tutkiminen voi syventää ymmärrystä funktionaalisen ohjelmoinnin periaatteista ja rakenteista. Kategoriateoria auttaa ymmärtämään funktioiden ja rakenteiden suhteita, mikä on keskeistä funktionaalisessa ohjelmoinnissa. Tämän ymmärryksen kautta voidaan saavuttaa tehokkaampia ja elegantimpia koodiratkaisuja, joissa turha toisto poistuu. Teorian läsnäolon tunnistaminen voi vahvistaa omia näkemyksiä toimimisesta, vaikka sen käyttö käytännössä ei olisi pakollista.
Bartosz Milewskin kirja \textquote*{Category Theory for Programmers} vaikuttaa erittäin lupaavalta tavalta opiskella kategoriateoriaa ohjelmoijan näkökulmasta \cite{milewski2017category}.

\subsection{Fantasyland-spesifikaatio}

Fantasyland-spesifikaatio on joukko sääntöjä algebrallisten rakenteiden, kuten monadien ja funktioiden, yhdistämiseksi JavaScriptissä \cite{fantasy-land:contributors}. Se perustuu kategoriateoriaan ja määrittelee, miten funktiot ja rakenteet vuorovaikuttavat, parantaen niiden yhteensopivuutta. Spesifikaatio sisältää esimerkiksi monadien \texttt{of} ja \texttt{bind} -operaatioiden säännöt. Fantasyland on erittäin teoreettinen ja määritelmällisesti tiukka, mikä kannustaa pedanttiseen funktionaaliseen ohjelmointiin, eroten tämän opinnäytetyön filosofiasta.

\subsection{Fp-ts-kirjasto}

Fp-ts-kirjasto auttaa puhtaan funktionaalisen ohjelmoinnin ymmärtämisessä TypeScript-ympäristössä. Kirjaston syntaksi poikkeaa perinteisestä TypeScript-koodista merkittävästi. Se on kuitenkin yksi parhaista vaihtoehdoista puhtaan funktionaalisen ohjelmoinnin opiskeluun suoraan TypeScriptissä \cite{holvikari2021category}.

\subsection{Eri ohjelmointikielet}
Eri ohjelmointikielet tarjoavat uusia näkökulmia funktionaaliseen ohjelmointiin. Haskellin tiukka tyyppijärjestelmä, pohja kategoriateoriassa, ja puhtaat funktiot opettavat funktionaalisen ohjelmoinnin keskeisiä periaatteita puhtaasti \cite{haskellallmonad,haskellcomposition,haskellmonadlaws}. Elixir yhdistää funktionaalisuuden ja ohjelman suorituksen rinnakkaisuuden, mikä auttaa ymmärtämään tehokasta moniydinkäsittelyä \cite{elixir}. Go-ohjelmointikielen opiskelu voisi myös muuttaa merkittävästi ajattelua ohjelmoinnista \cite{golang}. Go ei ole funktionaalinen ohjelmointikieli, vaikka se tukee joitain funktionaalisen ohjelmoinnin käsitteitä, kuten \glsdisp{higher_order_function}{korkeamman asteen funktioita}. Go:n käytännönläheisyys osoittaa, miten funktionaalisia periaatteita voi soveltaa tai jättää soveltamatta \cite{golang}.

\subsection{Tunnetut ohjelmistokehittäjät}

Funktionaalisen ohjelmoinnin ymmärtäminen vaatii monipuolista näkökulmaa eri lähteistä. Eri ohjelmistokehittäjien työ tarjoaa arvokkaita esimerkkejä ja oivalluksia, jotka rikastuttavat ymmärrystä funktionaalisten periaatteiden soveltamisesta ohjelmistokehityksessä.

Esimerkiksi Rich Hickey (Clojure-kielen kehittäjä) on tunnettu hyvin artikuloiduista ja mielekkäistä puheistaan ohjelmointifilosofioista funktionaaliseen ohjelmointiin liittyen \cite{hickey_maybe_not,hickey_persistent_2009}. Jake Archibald on ylläpitänyt HTTP 203 -sarjaa, nykyään Off the Main Thread -podcastia, sekä pragmaattista blogia, joissa hän käsittelee web-teknologioita vankalla kokemuksella \cite{against_self_closing_tags,is_reduce_bad}. Richard Feldman, tunnettu Elm-kielen puolestapuhuja, on osallistunut keskusteluihin funktionaalisesta ohjelmoinnista ja pitänyt puheita sen periaatteista. Hän korostaa ennustettavuutta ja virheiden vähentämistä, erityisesti käyttöliittymäkehityksessä \cite{feldman_fp_pragmatists,impossiblebetter,stackoverflow:why_monad}.

Opiskelemalla esimerkiksi näiden kehittäjien töitä löytyy arvokkaita oivalluksia ohjelmistokehitykseen niin funktionaalisen ohjelmoinnin, kun muun ohjelmoinnin puolelta. Heitä yhdistää pragmaattisuus. Ei turhaa teoriaa ilman käytäntöön perustuvaa perustelua.
