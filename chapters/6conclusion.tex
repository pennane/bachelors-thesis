% Material and Methods
%\clearpage%if the chapter heading starts close to bottom of the page, force a line break and remove the vertical vspace
\vspace{21.5pt}
\chapter{Johtopäätökset}

Funktionaalisen ohjelmoinnin periaatteet voidaan integroida perinteisempään ohjelmointiin. Liiallinen pedanttisuus käsitteitä kohtaan kuitenkin saattaa johtaa ohjelman ylläpidettävyyden ongelmiin, erityisesti muuttuvan koodin tuottamisessa asiakaskäyttöön. Ongelmat tulee mallintaa oikein, sillä tämä vähentää tarpeettomien testausvaiheiden ja if-lauseiden tarvetta. Hyvin mallinnettuna ohjelmakoodin määrä vähenee ja ylläpidettävyys paranee.

Akateemiset käsitteet funktionaalisessa ohjelmoinnissa ovat tiukasti määriteltyjä, mutta niiden noudattaminen ei vaadi pilkuntarkkaa tarkkuutta. Sokeaa sääntöjen noudattamista ilman perusteltua syytä tulee välttää. Yhdistetyt funktiot, funktioiden puhtaus, tiedon muuttumattomuus, algebralliset tietotyypit ja rakenteet ovat toimivia työkaluja ohjelmien rakentamisessa. Jos ohjelmointikieli ei tue niitä natiivisti, niiden tuominen mukaan saattaa heikentää koodin luettavuutta. On kuitenkin tärkeää kunnioittaa määritettyjä käytänteitä; kokemusten mukaan keskinkertaisetkin käytänteet ovat parempia kuin niiden puuttuminen kokonaan.

\section{Jatkotutkimus}

Funktionaalinen ohjelmointi on laaja ja monimuotoinen alue, joka kattaa teoreettisia käsitteitä ja käytännön sovelluksia. Jatkotutkimus voisi käsittää useita näkökulmia, ohjelmointikieliä ja tunnettujen ohjelmoijien julkaisuja. Moniulotteinen lähestymistapa on suositeltavaa, sillä pelkkä funktionaalisen ohjelmoinnin periaatteiden tutkiminen ei välttämättä riitä pragmaattisten ohjelmointitaitojen kehittämiseen.

\subsection{Kategoriateoria}

Kategoriateorian tutkiminen voi syventää ymmärrystä funktionaalisen ohjelmoinnin periaatteista ja rakenteista. Kategoriateoria auttaa ymmärtämään funktioiden ja rakenteiden suhteita, mikä on keskeistä funktionaalisessa ohjelmoinnissa. Tämän ymmärryksen kautta voidaan saavuttaa tehokkaampia ja elegantimpia koodiratkaisuja, joissa turha toisto poistuu. Teorian läsnäolon tunnistaminen voi vahvistaa omia näkemyksiä toimimisesta, vaikka sen käyttö käytännössä ei olisi pakollista.

\subsection{Fantasyland-spesifikaatio}

Fantasyland-spesifikaatio on määritelty joukko vaatimuksia ja sääntöjä, jotka koskevat funktioiden yhdistämistä ja algebrallisten rakenteiden käyttöä JavaScriptissä \cite{fantasy-land:contributors}. Se perustuu kategoriateorian perusperiaatteisiin, erityisesti algebrallisiin rakenteisiin, kuten monadeihin ja funktioihin. Spesifikaatio määrittelee, miten eri funktiot ja rakenteet voivat vuorovaikuttaa, tarjoten kehittäjille selkeät säännöt ja odotukset toiminnallisuuden implementoinnille. Tämä parantaa erilaisten funktioiden ja datarakenteiden yhdistämisen johdonmukaisuutta ja yhteensopivuutta.

Esimerkiksi spesifikaatiossa määritellään monadeille ja niiden yhdistämiselle liittyvät säännöt, kuten \texttt{of} ja \texttt{bind} -operaatiot, jotka ovat keskeisiä funktioiden yhdistämisessä. Fantasylandin periaatteiden ymmärtäminen voi auttaa kehittäjiä rakentamaan ohjelmia, joissa hyödynnetään funktionaalisen ohjelmoinnin parhaita käytäntöjä, mikä parantaa koodin luettavuutta ja ylläpidettävyyttä.

\subsection{Fp-ts-kirjasto}

Fp-ts-kirjaston tutkiminen auttaa puhtaan funktionaalisen ohjelmoinnin ymmärtämisessä TypeScript-ympäristössä. Vaikka fp-ts-kirjasto eroaa syntaksiltaan tavanomaisesta TypeScript-ohjelmointikoodista, se on todennäköisesti yksi parhaista vaihtoehdoista puhtaan funktionaalisen ohjelmoinnin opiskeluun TypeScriptissä \cite{holvikari2021category}.

\subsection{Eri ohjelmointikielet}

Eri ohjelmointikielten tutkiminen voi avata uusia näkökulmia funktionaalisen ohjelmoinnin käytäntöihin eri konteksteissa, sillä kielillä on omat erityispiirteensä. Esimerkiksi Haskell on yksi puhtaimmista funktionaalisista ohjelmointikielistä, ja sen tiukka tyyppijärjestelmä pakottaa kehittäjät ajattomaan ajatteluun ja puhtaiden funktioiden käyttöön \cite{haskellallmonad,haskellcomposition,haskellmonadlaws}. Haskellin kautta voi vahvistaa ajatuksia keskeisistä funktionaalisen ohjelmoinnin käsitteistä.

Toinen esimerkki on Elixir, moderni kieli, joka yhdistää funktionaalisen ohjelmoinnin periaatteet ja ohjelmistojen rinnakkaisuuden \cite{elixir}. Elixirin kyky käsitellä rinnakkaisia prosesseja auttaa opiskelijoita ymmärtämään, miten funktionaaliset ohjelmat voivat toimia tehokkaasti moniydinympäristöissä, korostaen samalla ohjelmien luettavuutta ja ylläpidettävyyttä.

Go-ohjelmointikielen opiskelu voisi myös muuttaa merkittävästi ajattelua ohjelmoinnista \cite{golang}. Vaikka Go ei ole funktionaalinen ohjelmointikieli, se tukee funktionaalisia periaatteita, kuten korkeampia funktioita. Go:n käytännönläheinen lähestymistapa voi auttaa ymmärtämään, miten funktionaalisia periaatteita voidaan soveltaa ohjelmoinnissa.

\subsection{Tunnetut ohjelmistokehittäjät}

Funktionaalisen ohjelmoinnin ymmärtäminen vaatii monipuolista näkökulmaa eri lähteistä, sillä se on laaja alue teoreettisista käsitteistä käytännön sovelluksiin. Eri ohjelmistokehittäjien työ tarjoaa arvokkaita esimerkkejä ja oivalluksia, jotka rikastuttavat ymmärrystä funktionaalisten periaatteiden soveltamisesta ohjelmistokehityksessä.

Rich Hickeyn, Jake Archibaldin ja Richard Feldmanin työt voivat avartaa näkemystä eri ohjelmointikielten erityispiirteistä ja niiden vaikutuksesta ohjelmistokehitykseen. Rich Hickey on Clojure-ohjelmointikielen kehittäjä, ja hänen työnsä korostaa ohjelmoinnin yksinkertaisuutta ja tehokkuutta \cite{hickey_maybe_not,hickey_persistent_2009}. Hickeyn filosofia, joka perustuu datan muuttumattomuuteen ja puhtaisiin funktioihin, tarjoaa syvällisen ymmärryksen siitä, miten ohjelmoinnin rakenteet tukevat luettavuutta ja ylläpidettävyyttä.

Jake Archibald tunnetaan erityisesti web-teknologioiden parissa työskentelevänä kehittäjänä \cite{against_self_closing_tags,is_reduce_bad}. Hänen näkemyksensä funktionaalisesta ohjelmoinnista, erityisesti JavaScriptin kontekstissa, auttaa opiskelijoita ymmärtämään, kuinka funktionaalisia periaatteita voidaan soveltaa käytännön projekteissa. Archibaldin työ tarjoaa käytännön esimerkkejä koodin laadun ja tehokkuuden parantamisesta web-kehityksessä.

Richard Feldman on Elm-ohjelmointikielen puolestapuhuja, ja hänen työnsä keskittyy käyttöliittymäkehitykseen \cite{feldman_fp_pragmatists,impossiblebetter}. Feldmanin näkökulma korostaa funktionaalisen ohjelmoinnin etuja, kuten ennustettavuutta ja virheiden vähentämistä, mikä on erityisen hyödyllistä suurissa ohjelmistoprojekteissa. Hänen esimerkit auttavat opiskelijoita ymmärtämään, kuinka funktionaaliset periaatteet voivat parantaa sovellusten laatua ja käyttäjäkokemusta.

Yhteenvetona, näiden kehittäjien työn tutkiminen tarjoaa käytännön esimerkkejä ja syvällisiä oivalluksia, jotka kasvattavat ymmärrystä ohjelmoinnista.

\section{Loppusanat}

Työn alkuvaiheilla olin vakuuttunut siitä, että funktionaalinen ohjelmointi on merkittävä etu. Nyt näkemykseni on, että funktionaalinen ohjelmointi on usein enemmän epäsuora voitto, eikä sitä tule pakottaa jokaiseen tilanteeseen.

Ajatukset ovat laajentuneet, ja on opittu ymmärtämään ohjelmoinnin monimuotoisuutta. Ohjelmointia tehdään yhteistyössä, jolloin yhteisiä käytäntöjä on syytä noudattaa.

On löydetty käytänteitä, jotka tulevat näkymään omissa tulevaisuuden ohjelmointiprojekteissa, sekä käytänteitä, joiden käyttöönottamista kannattaa arvioida tarkasti.

