% Introduction

\chapter{Johdanto}


Insinöörityössä tutkitaan funktionaalisen ohjelmoinnin käytänteitä sellaisissa ohjelmointikielissä, joissa funktionaalinen ohjelmointi ei ole ollut kielen luonnin peruste. Etsitään pragmaattista, tavoitelähtöistä puolta funktionaalisesta ohjelmoinnista, jossa teoriaa ei painoteta ilman selkeitä ja perusteltuja syitä. Funktionaalinen ohjelmointi on perusteiltaan erittäin teoreettista ja akateemisista lähtökohdista.

Kokemusten ja median tarkkailun perusteella on huomattu funktionaalisten periaatteiden yhdistyvän yleiseen näkemykseen siitä, mitä ohjelmoinnin tulisi olla. Funktionaalinen ohjelmointi ei ole kuitenkaan noussut valtavirran suosioon. Insinöörityön motivaationa on etsiä toimintaperiaatteita funktionaalisen ohjelmoinnin tuontiin mukaan arkiseen ohjelmointiin ilman pakonomaista teoriaa. Insinöörityössä kuitenkin etsitään funktionaalisen ohjelmoinnin teoriasta niitä osia, joilla voitaisiin kasvattaa funktionaalisen ohjelmoinnin pragmaattista mukaanottoa.

Insinöörityön koodiesimerkeissä käytetään \glsdisp{ts}{TypeScriptiä} ja \glsdisp{js}{JavaScriptiä}. Vaikka kaikki validi JavaScript koodi on validia TypeScript koodia \cite{typsecript_website}, niin kielten nimiä käytetään työssä tilanteeseen sopien. Kirjoitetaan TypeScriptistä, kun sen tarjoamat tyypit JavaScriptin päälle ovat tilanteeseen nähden merkittäviä. Toisaalta kirjoitetaan JavaScriptistä, kun tyypit eivät ole esimerkille merkittäviä. Muuten pyritään pitämään käsitellyt asiat sellaisina, että niitä voisi hyödyntää myös muissa ei-funktionaalisissa ohjelmointikielissä.