% Material and Methods
%\clearpage%if the chapter heading starts close to bottom of the page, force a line break and remove the vertical vspace
\vspace{21.5pt}
\chapter{Funktionaalisia käytänteitä ei-funktionaalisissa ohjelmointikielissä}

Tässä osassa pyritään esittämään kaikista oleellisin, mitä funktionaalisella ohjelmoinnilla voidaan saada aikaan. Teoreettisen perustan ja kokemusten avulla tuodaan esille mitä funktionaalinen ohjelmointi on koetusti parhaimmillaan.

Ajatuksena on, että kun ohjelma on mallinnettu oikein, ei ole suurta merkitystä ohjelmoidaanko olio-ohjelmointia, imperatiivista ohjelmointia, vai funktionaalista ohjelmointia. Olio-ohjelmoinnissa mallinnus saatettaisiin hoitaa rajapinnoilla (interface). Funktionaalisen ohjelmoinnin tapa mallintaa on käyttää tyyppejä ja abstrakteja (algebrallisia) tietorakenteita.


Tuodaan siis ilmi millaisia käytänteitä voidaan ripotella ohjelmointiin, vaikka ohjelmointikieli ei varsinaisesti olisi funktionaalinen.

Pyritään herättämään ajatuksia milloin ja miten funktioita kannattaa käyttää. Esitetään myös kuinka suuressa roolissa ongelmien mallintaminen on. Kun ongelmien mallit pureskellaan oikeisiin joukkoihin, saadaan turvallisempaa ohjelmakoodia. Silloin on myös helpompi kirjoittaa deklaratiivista koodia, joka on funktionaaliselle ohjelmoinnin ominaista.



\section{Opi kerran, käytä kaikkialla}

Funktionaalinen ohjelmointi kannustaa asioiden uudelleenkäytettävyyteen. Funktionaalisessa ohjelmoinnissa etsitään yleisiä teemoja, ja viedään toistuvia malleja funktioiksi, jotta voidaan kirjoittaa funktioiden nimiä syntaksin sijasta.


Esimerkiksi olio-ohjelmoinnissa, tai muuten imperatiivisessa ohjelmoinnissa, for-silmukalla toteutettavat algoritmit voidaan yleistää funktioiksi (Kuva \ref{fig:ramdacmds}). Onko helpompaa opetella ääretön määrä funktioita, vai yksi kielirakenne: for-silmukka?

\begin{figure}[ht]
    \centering
    \[
        \begin{array}{rl}
            \left.
            \begin{array}{l}
                filter, without, find, findIndex, findLast, findLastIndex \\
                map, mapIndexed, mapIndexedRight, chain, concat, zip      \\
                reduce, reduceRight, scan, partition, uniq, forEach       \\
                slice, drop, take, dropWhile, takeWhile, zip, zipWith     \\
                all, any, none
            \end{array}
            \right] \quad \text{for loop}
            \\
        \end{array}
    \]
    \caption{Ramda.js-kirjaston funktioita listojen käsittelyyn ohjelmoinnissa \cite{ramda:docs}. Kaikki nämä voitaisiin korvata for-silmukoilla imperatiivisessa paradigmassa}
    \label{fig:ramdacmds}
\end{figure}

Nimet ovat syntaksiriippumattomia. Samoja nimiä voi käyttää ohjelmointikielestä riippumatta. Nimillä voi kuvata oikean maailman asioita. Siksi voi perustella, että kun tietyn logiikan aina laittaa tietyn nimen taakse funktioksi, ymmärtää sen aina vaikka ohjelmointikieli vaihtuisi. Jos muistaa idean, ei implementaation yksityiskohdilla ole merkitystä.

Jos asia on tarpeeksi monimutkainen ja sen piilottaa nimen taakse, hyötyarvon määrittäminen voi olla haastavaa. Mitä monimutkaisempaa asiaa yritetään enkoodata yksittäiseen nimeen, sitä vaikeampaa se on lukijan ymmärtää, ja myös sitä vaikeampaa sille on löytää uudelleenkäyttötilanteita. Pienikin nyanssiero funktion implementaatiossa totuttuun implementaatioon voi mitätöidä opitun hyödyn eri ympäristöissä.

Yksinkertaisiakin funktioita voi olla vaikea hahmottaa koodista. Mikään ei pakota ohjelmoijaa nimeämään funktiota samalla tavalla kuin miten sama funktio on nimetty jossain toisessa projektissa tai ohjelmointikielessä. Nimeämiskäytänteiden noudattaminen ei kuitenkaan ole helppoa. Käytänteitä voi olla vaikea löytää, tai niitä voi olla useita \cite{ramda:docs}.

Kannattanee käyttää ohjelmointikielen sisäänrakennettuja funktioita, ja metodeja aina kun mahdollista. Näin voi maksimoida sen, että koodi ymmärretään. Jos kyse on jostakin projektin sisäisestä kirjastosta, voi olla hyödyllistä lisätä funktiolle kommentti, jos se tunnetaan muissa ohjelmointiympäristöissä jollain toisella nimellä.

Jos on kyse jostain toimintaympäristökohtaisesta funktiosta, eikä perustavanlaatuisesta yleispätevästä funktiosta, on tärkeää, että funktio nimetään mahdollisimman kuvaavasti. Esimerkiksi sen sijaan, että nimeää funktion nimellä \textquote{processUsers}, voi miettiä tarvitseeko funktion kutsupaikoissa ymmärtää enemmän siitä mitä funktio tekee? Jos kontekstista on selvää, mitä funktio tekee, nimen voi jättää sellaiseksi kuin on. Jos konteksti ei avaa toimintaa enemmän, \textquote{processUsers} on erittäin ympäripyöreä nimi.

\section{Yhdistetyt funktiot}

Kun funktioita on saatu kasattua työkalupakkiin, niitä voi yhdistellä erilaisten tarpeiden mukaan. Kerran kirjoitetut funktiot ovat monikäyttöisiä ja niitä voidaan ketjuttaa keskenään, jolloin monimutkaisemmat operaatiot rakentuvat valikoiduista palikoista.

Yhdistettyjen funktioiden avulla voidaan saavuttaa korkea taso ohjelmoinnin joustavuudessa. Ad hoc -funktioiden, eli tilanteeseen nopeasti mukautettujen funktioiden, kirjoittaminen on vaivatonta, ja niiden lisääminen osaksi olemassa olevia ketjuja onnistuu helposti. Tämä mahdollistaa nopean iteroinnin ja mukautumisen uusiin tarpeisiin ilman merkittäviä muutoksia alkuperäiseen koodiin.

Yhdistetyt funktiot eivät kuitenkaan automaattisesti takaa koodin luettavuutta. Liiallinen abstraktio voi tehdä koodista vaikeasti seurattavaa, erityisesti jos funktion yhdisteleminen tuo mukanaan monimutkaisia käsitteitä, jotka eivät ole suoraan yhteydessä ratkaistavaan ongelmaan. Turha abstraktio voi johtaa tilanteeseen, jossa koodin toiminnan ymmärtämiseksi täytyy perehtyä moniin välivaiheisiin ja ylimääräisiin konsepteihin, vaikka ongelma itsessään olisi yksinkertainen.

Esimerkkikoodi \ref{code:js_garbage_pipe} näyttää tilanteen, jossa Ramda.js-kirjastoa on käytetty yhdistetyn funktion rakentamiseen. Vaikka funktio toimii, sen ymmärtämiseen tarvitsee paljon ylimääräistä tietoa.

\begin{code}
    \begin{minted}{typescript}
const convertUsersToLeaderboardUsers = (users: Array<User>, anonymizeUsers: boolean) => R.pipe(
    R.pick(["userName", "score", "id"]),
    R.ifElse(
        R.always(anonymizeUsers), 
        R.map(R.omit(["userName"])), 
        R.identity
    ),
    R.map(R.assoc("type", "leaderboardUser")),
    R.sortWith([R.descend(R.prop("score"))])
)(users)
\end{minted}
    \caption{Funktio, muuttaa listan käyttäjiä sellaisiksi, että niitä voidaan käyttää tuloslistoilla. Funktio on yhdistetty monesta funktiosta käyttämällä Ramda.js-kirjaston funktioita. Funktiossa käytetään useita muita funktioita, joiden toiminnan lukijan on tiedettävä tai arvattava}
    \label{code:js_garbage_pipe}
\end{code}

Sen sijaan esimerkkikoodissa \ref{code:js_legit_pipe} käytetään vain JavaScriptin sisäänrakennettuja työkaluja. Funktio on yleisemmin helppo lukea. Vaikka luettavuus on subjektiivista, ja perustuu pitkälti ohjemointityylin tuttavuuteen, on todennäköisempää, että ohjelmoija ymmärtää JavaScriptin omia Array-metodeja, kuin minkäkin kirjaston funktioita, vaikka kirjaston funktiot oltaisiinkin pyritty kirjoittamaan noudattaen yleisiä käytänteitä.

\begin{code}
    \begin{minted}{typescript}
const convertUsersToLeaderboardUsers = (users: Array<User>, anonymizeUsers: boolean) => users
    .map(user => ({
        type: "leaderboardUser",
        userName: anonymizeUsers 
            ? undefined 
            : user.userName,
        score: user.score
    }))
    .sort((userA, userB) => userB.score - userA.score)
    \end{minted}
    \caption{Toiminnaltaan sama funktio, mutta käytettynä on vain JavaScriptin sisäänrakennettuja palikoita. Useampi lukija ymmärtää funktion, sillä se ei käytä ulkoisia kirjastoja}
    \label{code:js_legit_pipe}
\end{code}


On tärkeää, että abstrahoinnin taso pysyy hallittuna, eikä yhdistetty funktio tuo mukanaan lisämonimutkaisuutta. Abstraktioiden on oltava perusteltuja ja niiden tulee aidosti parantaa koodin ylläpidettävyyttä ja uudelleenkäytettävyyttä, eikä vain noudattaa jotain häilyviä teoreettisia malleja ilman konkreettisia hyötyjä.


Pitää siis pitää mielessä ohjelmointikieli, jossa toimitaan. Säännöt tulee sopeuttaa ympäristöön. Vaikka funktionaalisessa ohjelmoinnissa pyritään olla mutatoimatta dataa, tästä ei missään nimessä ole pakko pitää kiinni kynsin hampain.

\section{Puhdas funktio ei ole aina paras ratkaisu}

Funktionaalisessa ohjelmoinnissa pyritään välttämään mutatointia, eli ajonaikaisten muuttujien arvojen muuttamista. Tietorakenteiden muuttamisen sijasta pyritään luomaan tietorakenteista uusia kopioita, ja jättämään vanhat ennalleen \cite{immutablejs_immutable,hickey_persistent_2009}.

Joissain ohjelmointikielissä, tai ohjelmointikielien kirjastoissa, muuttumattomien tietorakenteiden käyttäminen on saatu varsin tehokkaaksi \cite{hickey_persistent_2009,immutablejs_immutable}. Näin ei kuitenkaan ole asian laita JavaScriptissä, tai ohjelmointikielissä yleisesti. Tiedon mutatointi on nopeaa ja kopiointi hidasta \cite{turner_trauring_copying_2020}. Näin ollen, jos kirjoitetaan funktioita, jotka palauttavat aina uusia kopioita, vanhojen mutatoimisen sijasta, saadaan usein hidasta ja tehotonta ohjelmakoodia.

Jossain vaiheessa ohjelman suoritusta tietokoneen bitit kääntyvät nollista ykkösiin, tai toisin sanoen mutatoituvat \cite{is_reduce_bad}. Niin tietokoneet toimivat, minkä vuoksi mutaatiota ei saa pelätä, kunhan sen hoitaa asianmukaisesti.

Jos funktio mutatoi ainoastaan muuttujia, joita se itse luo funktion ajon ajaksi, ei ole koodin turvallisuuden kannalta toiminnallista merkitystä mitä funktion sisällä tapahtuu. Kuitenkin tehokkuusvoitot voivat olla huomattavat versioihin, joissa mutatointia ei harjoiteta kuten seuraavat esimerkit osoittavat.

Esimerkkinä funktio, joka ottaa sisään listan avain-arvo-pareja, ja luo niistä olion, josta voi hakea arvoja avainten perusteella. Näytetään kaksi (funktionaalista) tapaa toteuttaa tämä mielivaltainen funktio.
Versio, joka ei mutatoi muuttujia missään vaiheessa (Koodiesimerkki \ref{code:js_just_mutate_immutable}), ja versio, joka mutatoi vain itse luomiaan muuttujia (Koodiesimerkki \ref{code:js_just_mutate_mutable}).

Tarkkailijan näkökulmasta molemmat funktiot ovat kuitenkin puhtaita.

\begin{code}
    \begin{minted}[highlightlines={8,14}]{javascript}
const entries = Array.from({ length: 1000 }, (_, i) => [
    `key-${i}`,
    `value-${i}`
])

const objectFromEntries = (entries) =>
    entries.reduce((acc, entry) => {
        return { ...acc, [entry[0]]: entry[1] }
    }, {})

console.time('Immutable')
objectFromEntries(entries)
console.timeEnd('Immutable')
// Immutable: 112.718017578125 ms
\end{minted}
    \caption{Funktio, joka ottaa listan avain-arvo-pareja ja luo niistä olion. Olion luonnissa ei käytetä ollenkaan mutatointia Funktion suorittaminen 1000:lle avain-arvo-parille kesti n. 113 millisekuntia}
    \label{code:js_just_mutate_immutable}
\end{code}

Täysin mutatoimattomalla versiolla kesti luoda tuhannesta avaimesta olio yli 300 kertaa kauemmin, kuin versiolla, joka mutatoi käyttämiään muuttujia (113 ms ja 0.4 ms).


\begin{code}
    \begin{minted}[highlightlines={8,15}]{javascript}
const entries = Array.from({ length: 1000 }, (_, i) => [
    `key-${i}`,
    `value-${i}`
])

const objectFromEntries = (entries) =>
    entries.reduce((acc, entry) => {
        acc[entry[0]] = entry[1]
        return acc
    }, {})
          
console.time('Mutable')
objectFromEntries(entries)
console.timeEnd('Mutable')
// Mutable: 0.35693359375 ms
\end{minted}
    \caption{Sama funktio kuin aiempi. Ainoa ero, että pareja iteroidessa luotavaa oliota mutatoidaan. Funktion suorittaminen 1000:lle avain-arvo-parille kesti vain n. 0.4 millisekuntia}
    \label{code:js_just_mutate_mutable}
\end{code}

Useimmiten ohjelman tehokkuudella ei ole paljoa merkitystä, ja luotettavuus ja luettavuus on syytä pitää etusijalla. Kuitenkin tässä tilanteessa tehokkuuserot ovat niin merkittävät, että jos 1 000:n avain-arvo-parin sijasta, olisi käytetty 100 000, tai vaikka 1 000 000 avain-arvo-paria, mutatoimaton versio ei välttämättä olisi suoriutunut tehtävästä koskaan. Näitä määriä yritettiin, mutta suoritukset lopetettiin kesken, kun alkoi näyttämään, ettei tehtävä tosiaan ollut suoriutumassa koskaan.

\section{Ongelmien mallintaminen lisää turvallisuutta}

Funktionaalinen ohjelmointi mahdollistaa deklaratiivisen ja itsensä selittävän koodin kirjoittamisen \cite{ms:fp_vs_imperative}. Oikeiden nimien löytäminen funktioille tai muille ohjelmiston osille on vaikeaa, jos ongelma on mallinnettu väärin. Tässä osiossa käsitellään pragmaattista mallinnusta pala palalta kuvitteelliselle alustalle.

Käydään läpi valheellisen mallin haitat ja oikean mallin hyödyt. Staattisen tyyppijärjestelmän avulla oikea malli poistaa ohjelmavirheitä jo käännösvaiheessa ja vähentää testaustarpeita \cite{impossiblebetter}.

Oikeellisen mallin avulla koodi on luotettavampaa, ja sen avulla voidaan poistaa tarpeettomia if-lauseita. Jos mallista poistetaan mahdottomat tilat, niitä ei tarvitse edes testata \cite{impossiblebetter}. Seuraavaksi muutetaan virheellinen malli oikeaksi.

Kuvitellaan, että ollaan rakentamassa rajapintaa verkossa toimivalle tietovisa-alustalle, jossa tietovisa koostuu useasta monivalintakysymyksestä. Mallinnus voi alkaa näin.

\begin{code}
    \begin{minted}{typescript}
type Question = {
  question: string
  answers: Array<string>
  correctAnswerIndices: Array<number>
}

type Quiz = {
  description: string
  questions: Array<Question>
}
         
\end{minted}
    \caption{Mahdollinen lähestymistapa yksinkertaiselle tietovisan mallinnukselle. Malli koostuu visasta (Quiz) ja se kysymyksistä (Question)}
    \label{code:ts_first_quiz}
\end{code}

Yksinkertaisen tietovisan saisi toimimaan kyseisellä mallilla. Yksi kysymys tukee jopa montaa oikeaa vastausta.
Mallintamisessa on kuitenkin ongelma. Se valehtelee. Sillä voidaan kuvata tiloja, joiden pitäisi olla mahdottomia.

\textbf{Listat voivat olla tyhjiä.} Malli ei pidä huolta, että tietovisassa on oltava vähintään yksi kysymys. Malli ei pidä myöskään huolta siitä, että kysymyksillä tulisi olla vähintään yksi vastaus. Tämä johtuu siitä, että \mintinline{typescript}|Array|-tietorakenne voi olla tyhjä.

Jos mallissa käytetään listaa, mutta se ei kuitenkaan todellisuudessa saisi olla tyhjä, niin ohjelmoijan on tarkistettava koodissa manuaalisesti, että listan sisältö on läsnä.

Vastauksena ongelmaan on luoda tyyppi listalle, joka yksiselitteisesti ei voi olla tyhjä (Koodiesimerkki \ref{code:ts_non_empty}).

\begin{code}
    \begin{minted}{typescript}
type NonEmptyArray<T> = {
    first: T
    rest: Array<T>
}
    \end{minted}
    \caption{Mahdollinen lähestymistapa yksinkertaiselle tietovisan mallinnukselle. Tyypissä on käytetty tyyppimuuttujaa \texttt{T}. Tyyppimuuttuja tarkoittaa sitä, että sen sijalle voi laittaa minkä tahansa tyypin. Kyse on myös parametrisestä polymorfista (parametric polymorphism).}
    \label{code:ts_non_empty}
\end{code}

Tyypillä on kenttä \mintinline{javascript}|first|, jonka arvon on oltava olemassa. Ja kenttä \mintinline{javascript}|rest|, jossa säilytetään listan loput alkiot. Tuon kentän lista taas on tavallinen \mintinline{javascript}|Array|. Pelkällä tyypin muutoksella voidaan pitää huolta että listassa on 0..n sijasta 1..n alkiota.

\begin{code}
    \begin{minted}[highlightlines={3,4,9}]{typescript}
type Question = {
    question: string
    answers: NonEmptyArray<string> 
    correctAnswerIndices: Array<number>
}

type Quiz = {
    description: string
    questions: NonEmptyArray<Question> 
}       
    \end{minted}
    \caption{Vaihtoehtoinen lähestymistapa tietovisan mallintamiselle, jossa käytetään itsemääritettyä NonEmptyArray-tyyppiä}
    \label{code:ts_non_empty_quiz}
\end{code}

Nyt tietovisassa on oltava vähintään yksi kysymys (Koodiesimerkki \ref{code:ts_non_empty_quiz}), ja jokaisella kysymyksellä vähintään yksi vastaus. Kuitenkin oikeita vastauksia ei ole välttämättä ainuttakaan, sillä kenttä \mintinline{typescript}|correctAnswerIncides| on edelleen tyypiltään \mintinline{typescript}|Array<number>|.

Staattisella tyyppijärjestelmällä voi pitää huolen, että mallia seurataan. Tarkastuksia tyhjyydestä ei tarvitse tehdä if-lauseilla. Malli ei kuitenkaan ole vieläkään täydellinen.

\textbf{Kenttien arvot voivat olla mitä tahansa.} Mikään ei pidä huolta, että esimerkiksi kentän \mintinline{typescript}|correctAnswerIncides| indeksit ovat \glsdisp{correctness}{oikeellisia}. Myös kaikki, joissa tyyppinä on \mintinline{typescript}|string| voivat ottaa vastaan mitä tahansa merkkijonoja. Myös tyhjiä merkkijonoja.

Merkkijonojen validointi TypeScriptin tyyppijärjestelmässä on kuitenkin hieman vaivalloista, eikä välttämättä tarpeellista. Kenttä \mintinline{typescript}|correctAnswerIncides| on kuitenkin ajatuksena erittäin huono, sillä malli ei millään tavalla määrittele miten se liittyy kysymykseen tarkemmin. Mikään ei estä laittamasta täysin päättömiä numeroita \textquote{oikeiksi} indekseiksi. Korjauskeinona on paras keino: kentän poistaminen.

Kentän voi poistaa kokonaan, jos mallinnusta viedään pidemmälle. Voidaan sopia, että on olemassa kahdenlaisia vastauksia. Oikeita tai vääriä. Tämä on varsin helppo mallintaa (Koodiesimerkki \ref{code:ts_answer_types})

\begin{code}
    \begin{minted}{typescript}
type CorrectAnswer = {
    type: "correct"
    answer: string
}
type IncorrectAnswer = {
    type: "incorrect"
    answer: string
}
    \end{minted}
    \caption{Oikeille ja väärille vastauksille omat tyypit}
    \label{code:ts_answer_types}
\end{code}

Uusilla tyypeillä voidaan mallintaa kysymyksiä \glsdisp{correctness}{oikeellisemmin}. Voidaan myös määrittää, että kysymyksellä on oltava vähintään yksi oikea vastaus, ja nolla tai useampi väärä vastaus (Koodiesimerkki \ref{code:ts_question_with_explicit_answer_types}).

\begin{code}
    \begin{minted}[highlightlines={3,4}]{typescript}
type Question = {
    question: string
    correctAnswers: NonEmptyArray<CorrectAnswer>
    incorrectAnswers: Array<IncorrectAnswer>
}
    \end{minted}
    \caption{Kysymykseen voi tarkentaa millaisia vastauksia hyväksytään}
    \label{code:ts_question_with_explicit_answer_types}
\end{code}

Malli ei enää valehtele. Voi kuitenkin olla, että tulevaisuudessa on tarve luoda visalle enemmän ominaisuuksia. Voidaan haluta rajoittaa, että kysymyksessä voi valita vain yhden vaihtoehdon, tai usean. Voidaan esimerkiksi haluta antaa mahdollisuus antaa vastauksena vapaata tekstiä \cite{impossiblebetter}. Muita tarpeita voi olla loputtomasti.

\textbf{Voidaan ottaa käyttöön summatyyppejä.} Sen sijasta, että lisättäisiin tyyppeihin uusia kenttiä kuten \mintinline{javascript}|isMultipleAnswerQuestion| tai \mintinline{javascript}|isFreeForm|, voidaan puolestaan luoda jokaiselle fundamentaalisesti erilaiselle asialle kokonaan oma tyyppinsä, ja yhdistää ne yhdeksi summatyypiksi. Näin pidetään huolta, että jokaisella asialla voi olla vain kenttiä, jotka kuuluvat sille.

\begin{code}
    \begin{minted}[highlightlines={20}]{typescript}
type MultipleAnswerQuestion = {
    type: "multipleAnswerQuestion"
    question: string
    correctAnswers: NonEmptyArray<CorrectAnswer>
    incorrectAnswers: Array<IncorrectAnswer>
}

type SingleAnswerQuestion = {
    type: "singleAnswerQuestion"
    question: string
    correctAnswer: CorrectAnswer
    incorrectAnswers: Array<IncorrectAnswer>
}

type FreeFormQuestion = {
    type: "freeFormQuestion"
    question: string
}

type Question = MultipleAnswerQuestion | SingleAnswerQuestion | FreeFormQuestion
    \end{minted}
    \caption{Kysymysten mallintaminen summatyypillä. Monivalintakysymykselle, yksinkertaiselle kysymykselle ja vapaatekstikentälliselle kysymykselle on jokaiselle oma tyyppi. Tyypit on nostettu yhteen summatyyppiin (Question)}
    \label{code:ts_sum_type_nice}
\end{code}

Kysymysten \mintinline{javascript}|type|-kentän perusteella voidaan koodissa tarkistaa millaisesta kysymyksestä on todella kyse, jonka jälkeen voidaan suorittaa koodia, joka liittyy vain kyseisen kysymyksen toimintaan.

Todellisuutta mallintavilla malleilla saadaan luotettavampaa koodia, ja poistetaan turhia if-lauseita. Myös jos mallista poistetaan mahdottomat tilat, niitä ei tarvitse edes testata.

\subsection{Virheet mukaan mallintamiseen}

Ohjelmissa tapahtuu virheitä. Vaikka pyrittäisiin malleihin, joista on poistettu mahdottomat tilat, niin ohjelmia ajaessa tullaan silti törmäämään virheisiin.

Miten on hyvä mallintaa tilanteita, kun virheitä tapahtuu?

Yleisesti imperatiivisissa paradigmoissa virhetilanteissa metodit ja funktiot palauttavat \texttt{null}-arvoja, virheitä tai poikkeuksia.
Funktionaalisessa ohjelmoinnissa voidaan käyttää samoja keinoja, mutta useimmin funktioiden suorituksen räjäyttämisen sijasta funktiot palauttavat summatyyppejä, tai monadeja, joissa voi olla sisällä joko oikea palautus- tai virhetyyppi.

TypeScriptissä yksinkertaisimmillaan tämän voi toteuttaa yhdellä omalla tyypillä (Koodiesimerkki \ref{code:ts_null_sum}).

\begin{code}
    \begin{minted}{typescript}
type Maybe<T> = T | null
const mightWorkMightNot = (): Maybe<string> => {
    if (Math.random() > 0.5) {
        return null
    }
    return "hello world"
}
    \end{minted}
    \caption{Mahdollisesti puuttuvan paluuarvon malli. Maybe voi sisältää jonkin tyypin tai arvon \texttt{null}}
    \label{code:ts_null_sum}
\end{code}

Deklaratiivisesti funktio kertoo, että se saattaa palauttaa merkkijonon, tai saattaa olla palauttamatta. Ehkä se on merkkijono, ehkä se ei ole.
Tyypin käyttäminen on hieman tönkköä, sillä jos funktion oikea palautusarvo on null, on mahdotonta sanoa oliko funktiokutsu onnistunut vai ei.

Tyypin voi viedä pidemmälle, jos arvot kääritään objekteihin. Vaihdetaan nimi Result-tyypiksi, että tyyppi voi kapseloida myös virheitä. (Koodiesimerkki \ref{code:ts_better_result_sum}.)

\begin{code}
    \begin{minted}{typescript}
type Success<T> = { value: T }
type Failure = { error: Error }
type Result<T> = Success<T> | Failure 

const mightWorkMightNot = (): Result<string> => {
    if (Math.random() > 0.5) {
        return { error: new Error("Sorry, bad failure.") }
    }
    return { value: "hello world" }
}
    \end{minted}
    \caption{Vaihtoehtoinen malli mahdollisesti epäonnistuvalle paluuarvolle. Result on joko Success, tai Failure. Molemmissa tapauksissa arvo on kuitenkin yhden kentän objekti, jonka ansiosta tyyppiin voi tallentaa \texttt{null}-arvon ilman että tietoa menetetään}
    \label{code:ts_better_result_sum}
\end{code}


Näin Result-tyyppi voi sisältää myös arvon \mintinline{typescript}|null|, sillä Result-objektin tarkan tyypin voi nyt päätellä ilman sitä. Nyt tarkan tyypin voi määrittää \mintinline{typescript}|value| tai \mintinline{typescript}|error| kentän läsnäolosta tai puutteesta. Näin rakennettuna tyypille on myös helppo rakentaa apufunktioita, joilla selvitetään, onko paluuarvo onnistunut, vai ei. Voi myös luoda apufunktiot, joilla luoda Result-objekteja. (Koodiesimerkki \ref{code:result_helpers}).

\begin{code}
    \begin{minted}{typescript}
const isFailure = <T>(x: Result<T>): x is Failure => 
    "error" in x
const isSuccess = <T>(x: Result<T>): x is Success<T> => 
    !isFailure(x)
const failureOf = (error: string): Failure => 
    ({ error: new Error(error) })
const successOf = <T>(value: T): Success<T> => 
    ({ value })
    \end{minted}
    \caption{Apufunktioita Result-tyypin tarkastamiseen ja luomiseen TypeScriptissä}
    \label{code:result_helpers}
\end{code}

Näillä voidaan rakentaa yhdistettyjä funktioita, joissa paluuarvon luonti ja palauttaminen on eksplisiittistä ja selkeää. Koodiesimerkissä \ref{code:result_pipe_example} käsitellään Result-objekteja yhdistetyssä funktiossa. Jokaisessa yhdisteen funktiossa tarkistetaan onko aiempi tulos epäonnistunut. Epäonnistumisen sattuessa virhearvo päästetään suoraan putken läpi.

\begin{code}
    \begin{minted}{typescript}
const helloMaybe = (): Result<string> => {
    if (Math.random() > 0.5) {
        return failureOf("Sorry, bad failure.")
    }
    return successOf("hello world")
}

const addExclamation = (x: Result<string>): Result<string> => {
    if (isFailure(x)) return x
    return successOf(x.value + "!")
}

const toUpperCase = (x: Result<string>): Result<string> => {
    if (isFailure(x)) return x
    return successOf(x.value.toUpperCase())
}

pipe([
    helloMaybe,
    addExclamation,
    toUpperCase
])()
// Palauttaa { value : "HELLO WORLD!}
// tai { error: Error("Sorry, bad failure.") }
    \end{minted}
    \caption{Esimerkki yhdistetystä funktiosta Result-tyypin kanssa.}
    \label{code:result_pipe_example}
\end{code}

Result-tyypin käyttö on selkeää, ja TypeScriptillä on pakko tarkistaa, onko arvo \mintinline{typescript}|Success| vai \mintinline{typescript}|Failure|, ennen kuin sen käärittyä arvoa voi tarkastella tai muokata.

Jos yksikin funktio palauttaa Result-objektin, jossa on virhe, objekti kulkee muuttumattomana loppujen funktioiden lävitse.

Kuitenkin ongelma on, että tämänkaltaisena Result-tyypin käyttäminen on loppujen lopuksi vain syntaksisokeroitu null-arvon tarkistus. Ne if-lauseet, joita aiemmin poistettiin todenmukaisella tyyppimallinnuksella, ovat tulleet takaisin aina pakollisen virhetarkistamisen muodossa.

Result-tyypillä on kuitenkin jotain, jolla sen voi viedä vielä askelta pidemmälle. Result-tyypin voi toteuttaa monadina.


\subsection{Result-monadi}

Monadi oli suuressa osassa teoriaosuudessa. Jatketaan mahdollisesti epäonnistuvan ohjelman toteuttamista siten, että siinä on mukana monadisia periaatteita. Monadia käyttämällä voidaan siirtää virheentarkastuksen if-lauseet yhteen paikkaan keskitetysti.

Luodaan \texttt{Result}-monadi (Koodiesimerkki \ref{code:result_monad_impl}).


\begin{code}
    \begin{minted}{typescript}
class Result {
    constructor(x) {
        this.value = x
    }
    static of(x) {
        return new Result(x)
    }
    bind(f) {
        if (this.value instanceof Error) {
            return this
        }

        const result = f(this.value)

        if (!(result instanceof Result)) {
            return Result.of(result)
        }

        return x
    }
}
\end{minted}
    \caption{Result implementoituna monadiksi. Result monadeja voi luoda staattisella \mintinline{js}|Result.of|-metodilla, ja monadiin voi ketjuttaa operaatioita \mintinline{js}|Result.prototype.bind|-metodilla. Jos ketjutuksessa monadin arvo on Error, niin ketjutus lopetetaan, ja saatu virhearvo kuljetetaan suoraan lävitse}
    \label{code:result_monad_impl}
\end{code}

Jos \texttt{Result}-monadi otetaan käyttöön aiemman osan koodiesimerkkiin, huomataan luettavan koodin määrän pienentyneen (Koodiesimerkki \ref{code:result_monad_example_2}).

\begin{code}
    \begin{minted}{typescript}
const helloMaybe = (): string | Error => 
    Math.random() > 0.5 
        ? "hello world" 
        : new Error("Sorry, bad failure.")

const addExclamation = (x: string) => x + "!"
const toUpperCase = (x: string) => x.toUpperCase()

Result.of(helloMaybe())
 .bind(addExclamation)
 .bind(toUpperCase)
\end{minted}
    \caption{Monadisia operaatioita ketjutettuna Result-monadiin. Annettavat funktiot voivat palauttaa käärimättömiä arvoja johtuen Result-monadin implementaatiosta}
    \label{code:result_monad_example_2}
\end{code}

Jos yksikin monadin funktioketjusta epäonnistuu, suoritus lakkaa ja virhe palautuu loppuun asti. Arvon voi vetää ulos monadista useilla eri tavoilla. Implementaation ansiosta tässä yhteydessä arvon voi ottaa ulos monadista yksinkertaisesti käyttäen \texttt{value}-kenttää (Koodiesimerkki \ref{code:result_monad_example_3})


\begin{code}
    \begin{minted}{typescript}
const monad = Result.of(helloMaybe())
    .bind(addExclamation)
    .bind(toUpperCase)
 const value = monad.value
 // value on joko "HELLO WORLD!" tai Error("Sorry, bad failure")
\end{minted}
    \caption{Arvon voi poistaa monadin kontekstista yksinkertaisesti hakemalla \mintinline{js}|value|-kenttää }
    \label{code:result_monad_example_3}
\end{code}

Eri ohjelmointikielissä tai kirjastoissa monadeista arvon ulos saaminen voi poiketa. Matemaattisessa määritelmässä arvoa ei ole tarkoitus koskaan saada pois monadista. Esimerkiksi Haskellissa lähtökohtaisesti arvo kuuluu aina jättää monadinsa kontekstiin, eikä sitä tulisi käsitellä suoraan \cite{haskellallmonad}.