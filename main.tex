%! TEX TS-program = xelatex
\documentclass[12pt,a4paper,oneside,article]{memoir}%Do not touch this first line ;)
%----------------------------------------------------------------------------------------
%	Metropolia Thesis LaTeX Template
%----------------------------------------------------------------------------------------

%----------------------------------------------------------------------------------------
%	THESIS INFO
%----------------------------------------------------------------------------------------

% All general information (main language, title, author (you), degree programme, major
% option, etc.)
% Edit the file chapters/0info.tex to change these information

% Global information (title of your thesis, your name, degree programme, major, etc.)
\def\bilingual{yes}
\def\thesislang{finnish}
\def\secondlang{english}
\author{Arttu Pennanen}
%License
\def\thesiscopy{by}

%Finnish section, for title/abstract
\def\otsikko{Pragmaattisen funktionaalisen \\ohjelmoinnin arviointi}
\def\tutkinto{Insinööri (AMK)}
\def\kohjelma{Tieto- ja viestintätekniikka}
\def\suuntautumis{Ohjelmistotuotanto}
\def\thesisfi{Insinöörityö}
\def\ohjaajat{
    FM Simo Silander
}
\def\tiivistelma{
    Funktionaalista ohjelmointia tukee vahva teoreettinen tausta, ja sen käytännön toteutus eroaa olennaisesti muista ohjelmointiparadigmoista. Tässä insinöörityössä tarkastellaan, miten funktionaalisen ohjelmoinnin periaatteita voidaan pragmaattisesti sisällyttää ohjelmistoprojekteihin.\newline

    Käydään läpi, miten ohjelmistoympäristöön voidaan tuoda funktionaalisen ohjelmoinnin kieliagnostisia piirteitä, jotka eivät ole sidoksissa käytettävään ohjelmointikieleen. Koodiesimerkit, näkemykset ja perustelut pohjautuvat kokemukseen ohjelmistoyrityksessä sekä julkisiin lähteisiin, joissa käsitellään funktionaalisen ohjelmoinnin, kategoriateorian, ongelmien mallintamisen ja ohjelmoinnin periaatteita.\newline

    Tuloksena perustellaan funktionaalisen ohjelmoinnin käyttöä ohjelmien oikeellisuuden ja ylläpidettävyyden parantamiseksi. Näytetään, miten ohjelmat voidaan estää valehtelemasta ja miten operaatioita voidaan ketjuttaa funktioiden ja monadien avulla. Funktionaalista ohjelmointia voidaan liittää muihin ohjelmointiparadigmoihin, kunhan huolehditaan ohjelmiston monimutkaisuuden hallinnasta. Ohjelmoijat lähtökohtaisesti ajattelevat ongelmia ja ratkaisuja sen ohjelmointikielen näkökulmasta, jota he käyttävät.
}
\def\avainsanat{funktionaalinen ohjelmointi, JavaScript, TypeScript, koodin luettavuus, koodin ylläpidettävyys, koodin suorituskyky, kehittäjäkokemus}
\def\aihe{Insinöörityössä tarkastellaan funktionaalisten periaatteiden soveltamista JavaScriptissä perustuen käytännön kokemukseen ja julkisiin lähteisiin. Työssä korostuu ohjelmien oikeellisuuden ja ylläpidettävyyden parantaminen funktioiden ja monadien avulla.}%for the pdf metadata/properties. If not used, empty it and also the \def\subject.

%English section, for title/abstract
\title{Evaluating Pragmatic Application of Functional Programming}
\def\metropoliadegree{Bachelor of Engineering} % change to your needs, e.g. "master", etc.
\def\metropoliadegreeprogramme{Information and Communication Technology}
\def\metropoliaspecialisation{Software Engineering}
\def\thesisen{Bachelor’s Thesis} % change to your need, e.g. master's
\def\metropoliainstructors{
    Simo Silander, M.Sc.
}
\def\abstract{
    Functional programming is underpinned by a robust theoretical foundation in mathematics, distinguishing it significantly from other programming paradigms. This thesis explores how the principles of functional programming can be pragmatically applied into software projects.\newline

    The center of discussion is introducing language-agnostic features of functional programming into the software environment. Code examples, insights, and justifications are drawn from experiences in a software company, and from public sources focusing on the principles of functional programming, category theory, problem modeling, and programming in general.\newline

    As a result, an argument is drawn in favor of functional programming to enhance the correctness and maintainability of programs. It is demonstrated how programs can be prevented from lying, and how operations can be chained using functions and monads. Functional programming can, and should, be integrated into other programming paradigms, provided that software complexity is taken care of. Programmers typically assume they are coding in the language they are using.
}
\def\metropoliakeywords{functional programming, JavaScript, TypeScript, code readability, code maintainability, code performance, developer experience}
\def\subject{This thesis explores the application of functional principles in JavaScript based on practical experience and public sources. It emphasizes improving the correctness and maintainability of programs through the use of functions and monads.}%for the pdf metadata/properties. If not used, empty it and also the \def\aihe.


%----------------------------------------------------------------------------------------
%	GLOBAL STYLES
%----------------------------------------------------------------------------------------

% If you need extra package, etc. modify the style/style.tex file.
% If you are using Windows OS, you will need to change default font to Arial in that
% style/style.tex file (or install Liberation Sans font to your system).
% If you are using MacOS or linux, make sure you have Liberation Sans font installed.
\input{style/style.tex}
% Normally, you do not need to modify the title style. It's content comes from the
% chapters/0info.tex file.
\input{style/title.tex}

%----------------------------------------------------------------------------------------
%	ABBREVIATION AND GLOSSARY
%----------------------------------------------------------------------------------------

% Add/edit all your acronyms, abbreviations, glossary entries, etc. definitions in
% chapters/0abbr.tex file.
% You can have as many as you wish. Only the ones you use in your text (inserted with
% \gls{} command) will print in the Glossary/Lyhenteet.
% Generate the glossary
\makeglossaries

% Acronyms, abbreviations, etc.

\newacronym{js}{js}{JavaScript}
\newacronym{io}{I/O}{Input/Output}
\newacronym{url}{URL}{Uniform Resource Locator}

% Glossary entries

\newglossaryentry{functional_programming}{
	name={funktionaalinen ohjelmointi},
	description={ohjelmointiparadigma, joka korostaa laskennan mallintamista funktioiden avulla. Funktioita käsitellään ensisijaisina rakennuspalikoina, ja ne ovat yleensä puhtaita eli ilman sivuvaikutuksia. Funktionaalinen ohjelmointi kannustaa käyttämään immuuttisia tietorakenteita ja väistämään tilan ja sivuvaikutusten käyttöä, mikä tekee ohjelmista ennustettavampia ja helpommin ymmärrettäviä.}
}





%----------------------------------------------------------------------------------------
%	DOCUMENT STARTS HERE...
%----------------------------------------------------------------------------------------

\begin{document}
\IfLanguageName{finnish}{
}{
  \raggedright%2021 template, align left, no hyphennization for English version
}
\counterwithout{listing}{chapter}

%----------------------------------------------------------------------------------------
%	TITLE PAGE
%----------------------------------------------------------------------------------------

\input{style/title_headers.tex}
\maketitle
\newpage

%----------------------------------------------------------------------------------------
%	ABSTRACT / Tiivistelmä
%----------------------------------------------------------------------------------------

% If you are international student writing in English, ignore the Finnish abstract.
% If you are Finnish citizen, you must have 2 abstracts, one in Finnish (or Swedish
% depending on your mother tongue) and one in English regardless of the main language of
% your thesis. Normally, you do not need to modify the abstract style. It's content comes
% from the chapters/0info.tex file.
\ifdefstring{\bilingual}{no}{%
  \input{style/abstract_en.tex}
}{%
  \IfLanguageName{finnish}{%order of abstracts based on main language and spacing hell
    % Abstract in Finnish
% Normally, you should not edit this file. Everything comes from chapter/0info.tex

\pagestyle{empty} %remove page number
\newgeometry{top=1.5cm,left=4cm,right=.5cm}
\begin{otherlanguage}{finnish}
  {\large\textbf{Tiivistelmä}}
  {\renewcommand{\arraystretch}{1.1}
    \begin{tabular}{@{}p{4.7cm} >{\raggedright\arraybackslash}p{10.8cm}@{}}
      Tekijä:               & \makeatletter\@author\makeatother
      \\
      Otsikko:              & \parbox[t]{10.8cm}{\otsikko}
      \\
      Sivumäärä:            & \pageref*{LastPage} sivua         %+ \total{chapter} liitettä
      \\
      Aika:                 & \pvm
      \\[6.5mm]
      Tutkinto:             & \tutkinto
      \\
      Tutkinto-ohjelma:     & \kohjelma
      \\
      Ammatillinen pääaine: & \suuntautumis
      \\
      Ohjaajat:             & \ohjaajat
      \\[11mm]
      \cmidrule[.7pt](l{-.15em}r{5.5em}){1-2}
      \multicolumn{2}{>{\raggedright}p{15.5cm}}{
      \vspace{.25mm}
      \makebox[0pt][l]{\hspace{-0.2cm}\parbox{\dimexpr\textwidth-1cm\relax}{\tiivistelma}}
      }
      \\
      \\
      Avainsanat:           & \avainsanat
      \\
      \\[1mm]
      \cmidrule[.7pt](l{-.15em}r{5.5em}){1-2}
      \\[0.2mm]
      \makebox[0pt][l]{\parbox{\dimexpr\textwidth\relax}{Tämän opinnäytetyön alkuperä on tarkastettu Turnitin Originality Check -ohjelmalla.}}
    \end{tabular}
  }
\end{otherlanguage}
\restoregeometry
\clearpage


    \input{style/abstract_fi_en.tex}
  }{
    \input{style/abstract_en.tex}
    \input{style/abstract_fi.tex}
  }
}
%----------------------------------------------------------------------------------------
%	License? Acknowledgement?
%----------------------------------------------------------------------------------------

% Uncomment next line and edit chapters/0license.tex if you want license in your thesis.
% License of your thesis
% If you wish to explain what it means. When you publish your thesis in https://theseus.fi
% you will be able to choose between some Creative Commons licenses
% https://creativecommons.org
% Adapt this example text to your taste.
% This would also be the right place to explain the license you choose for the code you
% produced for your thesis.

\pagestyle{empty}
\chapter*{Lisenssit}
\ifdefstring{\thesiscopy}{all}{}{%
  \begin{wrapfigure}{r}{0.3\textwidth}
    \vspace{-20pt}
    \doclicenseImage
  \end{wrapfigure}
 }
\IfLanguageName{finnish}{\copyfi}{\copyen}

Tämä tarkoittaa:

\textbf{Voit vapaasti:}
\begin{itemize}
\item Jakaa \textemdash kopioida aineistoa ja levittää sitä edelleen missä tahansa välineessä ja muodossa missä tahansa tarkoituksessa, myös kaupallisesti.
\item Muunnella \textemdash remiksata ja muokata aineistoa sekä luoda sen pohjalta uusia aineistoja missä tahansa tarkoituksessa, myös kaupallisesti.
\end{itemize}

\textbf{Seuraavilla ehdoilla:}
\begin{itemize}
\item Nimeä \textemdash Sinun on mainittava lähde asianmukaisesti, tarjottava linkki lisenssiin sekä merkittävä, mikäli olet tehnyt muutoksia. Voit tehdä yllä olevan millä tahansa kohtuullisella tavalla, mutta et siten, että annat ymmärtää lisenssinantajan suosittelevan sinua tai teoksen käyttöäsi.
\item Ei muita rajoituksia \textemdash Et voi asettaa sellaisia oikeudellisia ehtoja tai teknisiä estoja, jotka estävät oikeudellisesti muita tekemästä mitään sellaista, minkä lisenssi sallii.
\end{itemize}

%Eventually consider few words why you choose such license? E.g. something like:
% I decided to publish my thesis work under the Creative Commons Attribution-ShareAlike 4.0 International License because I strongly believe that you as reader deserve the freedom to copy, share and modify this work and if you do modify it, it is fair to give these same permissions to the others. A copy in electronic form of this work can be found in \url{https://some.place} with the \LaTeX{} source.

\clearpage


% Uncomment next line and edit chapters/0acknowledgement.tex if you want acknowledgements.
%\input{chapters/0acknowledgement.tex}

%----------------------------------------------------------------------------------------
%	TABLE OF CONTENTS
%----------------------------------------------------------------------------------------

\input{style/toc.tex}

%list of figure, tables would come here if relevant?

%----------------------------------------------------------------------------------------
%	Lyhenteet / Abbreviation
%----------------------------------------------------------------------------------------

% If you don't use abbreviations/glossary, remove the following line.
\input{style/abbr.tex}

%----------------------------------------------------------------------------------------
%	CONTENT
%----------------------------------------------------------------------------------------

\input{style/content.tex}%reset page number to 1, etc.

% Introduction

\chapter{Johdanto}

...

Insinöörityön koodiesimerkeissä käytetään \glsdisp{ts}{TypeScriptiä} ja \glsdisp{js}{JavaScriptiä}. Vaikka kaikki validi JavaScript koodi on validia TypeScript koodia, niin kielten nimiä käytetään työssä tilanteeseen sopien. Kirjoitetaan TypeScriptistä, kun sen tarjoamat tyypit JavaScriptin päälle ovat tilanteeseen nähden merkittäviä. Toisaalta kirjoitetaan JavaScriptistä, kun tyypit eivät ole esimerkille merkittäviä.

...
% Theoretical background
%\clearpage % Uncomment if needed to force a page break before the chapter
\vspace{21.5pt} % Consider removing unless extra space is absolutely necessary
\chapter{Teoreettinen tausta}

TODO - pelkkää filleriä ja testausta

da fak on kaan \gls{functional_programming}.

\section{Funktioiden yhdistäminen}

Olkoon \( f: A \rightarrow B \) ja \( g: B \rightarrow C \) kaksi funktiota. Funktioiden \( f \) ja \( g \) yhdistämistä merkitään \( g \circ f \), ja se on funktio joukosta \( A \) joukkoon \( C \), joka määritellään seuraavasti:

\[
  (g \circ f)(x) = g(f(x))
\]

Esimerkiksi, jos \( f(x) = 2x + 3 \) ja \( g(x) = x^2 \), niin yhdistetty funktio \( (g \circ f)(x) \) lasketaan seuraavasti:

\[
  (g \circ f)(x) = g(f(x)) = g(2x + 3) = (2x + 3)^2
\]

Laajennetaan \( (2x + 3)^2 \):

\[
  (g \circ f)(x) = 4x^2 + 12x + 9
\]

\section{Blaa ja blaa}

Imperatiivinen \ref{code:imperative} vai deklaratiivinen \ref{code:declarative}? Ketä ginee.

Imperatiivisestni:

\begin{code}
  \inputminted{javascript}{code/imperative.js}
  \captionof{listing}{Imperatiivinen tapa \gls{js} koodissa poistaa parilliset ja tuplata parittomat numerot}
  \label{code:imperative}
\end{code}

thö declarative:

\begin{code}
  \inputminted{javascript}{code/declarative.js}
  \captionof{listing}{Funktionaalinen tapa \gls{js} koodissa poistaa parilliset ja tuplata parittomat numerot}
  \label{code:declarative}
\end{code}


Siitä sitten sanomaan. Imperatiivinen \ref{code:imperative} on nopeampi kuin deklaratiivinen \ref{code:declarative}, mut luettavuus sitten toisin päin.

\section{Blaa blaa x 2}

Feldmanin mukaan, mahdottomat tilat ovat testausta parempaa \cite{impossiblebetter}.

\subsection{BLAA :D}
ja niin edelleen

% Material and Methods
%\clearpage%if the chapter heading starts close to bottom of the page, force a line break and remove the vertical vspace
\vspace{21.5pt}
\chapter{Empiirinen tutkimus}

TODO

\section{Tutkimuksen lähtökohdat}

TODO

\subsection{Tutkimusympäristö}

TODO

\subsection{Tutkimusmenetelmät}

TODO

\subsection{Tutkimuksen rajoitukset}

TODO


\section{Opi kerran, käytä kaikkialla}

Funktionaalisessa ohjelmoinnissa etsitään yleisiä teemoja, ja viedään toistuvia malleja funktioiksi, jotta voidaan syntaksin sijasta kirjoittaa funktioiden nimiä. Olio-ohjelmoinnissa, tai muuten imperatiivisessa ohjelmoinnissa, for-silmukalla toteutettavat algoritimit yleistetään funktioiksi (\ref{fig:ramdacmds}). Onko helpompaa opetella ääretön määrä funktioita, vai yksi kielirakenne: for-silmukka?

\begin{figure}[ht]
    \centering
    \[
        \begin{array}{rl}
            \left.
            \begin{array}{l}
                filter, without, find, findIndex, findLast, findLastIndex \\
                map, mapIndexed, mapIndexedRight, chain, concat, zip      \\
                reduce, reduceRight, scan, partition, uniq, forEach       \\
                slice, drop, take, dropWhile, takeWhile, zip, zipWith     \\
                all, any, none
            \end{array}
            \right] \quad \text{for loop}
            \\
        \end{array}
    \]
    \caption{Ramda.js kirjaston funktioita listojen käsittelyyn ohjelmoinnissa \cite{ramda:docs}. Kaikki nämä voidaan korvata for-silmukoilla imperatiivisessa paradigmassa.}
    \label{fig:ramdacmds}
\end{figure}

Nimet ovat syntaksiriippumattomia. Samoja nimiä voi käyttää ohjelmointikielestä riippumatta. Nimillä voi kuvata oikean maailman asioita. Siksi voi perustella, että kun tietyn logiikan aina laittaa tietyn nimen taakse funktioksi, ymmärtää sen aina vaikka ohjelmointikieli vaihtuisi. Jos muistaa idean, ei implementaation yksityiskohdilla ole merkitystä.

Jos asia on tarpeeksi monimutkainen ja sen piilottaa nimen taakse, hyötyarvon määrittäminen voi olla haastavaa. Mitä monimutkaisempaa asiaa yritetään enkoodata yksittäiseen nimeen, sitä vaikeampaa se on lukijan ymmärtää, ja myöskin sitä vaikeampaa sille on löytää uudelleenkäyttötilanteita. Pienikin nyanssiero funktion implementaatiossa totuttuun implementaatioon voi mitätöidä opitun hyödyn eri ympäristöissä.

Yksinkertaisiakin funktioita voi olla vaikea hahmottaa koodista. Mikään ei pakota ohjelmoijaa nimeämään funktiota samalla tavalla kuin miten sama funktio on nimetty jossain toissessa projektissa tai ohjelmointikielessä. Vaikka ohjelmoija tahtoisi noudattaa aiempia nimeämiskäytänteitä, niin aiempia nimeämiskäytänteitä voi olla vaikea löytää, tai ylipäätään jo useita.

Kannattanee käyttää ohjelmointikielen sisäänrakennettuja funktioita, ja metodeja aina kun mahdollista. Näin voi maksimoida sen, että koodi ymmärretään. Jos kyse on jostakin projektin sisäisestä kirjastosta, voi olla hyödyllistä lisätä funktiolle kommentti, jos se tunnetaan muissa ohjelmointiympäristöissä jollain toisella nimellä.

Jos on kyse jostain toimintaympäristökohaisesta funktiosta, eikä perustavanlaatuisesta yleispätevästä funktiosta, on tärkeää että funktio nimetään mahdollisimman kuvaavasti. Esimerkiksi sen sijaan, että nimeää funktion nimellä \textcite{processUsers}, voi miettiä tarvitseeko funktion kutsupaikoissa ymmärtää enemmän siitä mitä funktio tekee? Jos kontekstista on selvää, mitä funktio tekee, nimen voi jättää sellaiseksi kuin on. Jos konteksti ei avaa toimitaa enemmän, \textcite{processUsers} on erittäin ympäripyöreä nimi.

\section{Yhdistetyt funktiot}

Nimiä voi täydentää toisilla nimillä. :DDDDDDDDDDDDDDDDDDDD JEAA BOII


\section{Puhtaiden funktioiden hinta}

Pitää pitää mielessä ohjelmointikieli, jossa toimitaan. Säännöt tulee sopeuttaa ympäristöön. Vaikka funktionaalisessa ohjelmoinnissa pyritään olla mutatoimatta dataa, tästä ei missään nimessä ole pakko pitää kiinni kynsin hampain.

Joissain ohjelmointikielissä, tai ohjelmointikielien kirjastoissa, muuttumattomien tietorakenteiden käyttäminen on saatu varsin tehokkaaksi. Näin ei kuitenkaan ole asian laita JavaScriptissä, tai ohjelmointikielissä yleisesti. Yleisesti ottaen tiedon mutatointi on nopeaa ja kopiointi hidasta. Näin ollen, jos kirjoitetaan funktioita, jotka palauttavat aina uusia kopioita, vanhojen mutatoimisen sijasta, saadaan usein hidasta ja tehotonta ohjelmakoodia.

Jossain vaiheessa ohjelman suoritusta tietokoneen bitit kääntyilevät nollista ykkösiin. Näin tietokoneet toimivat. Tämän vuoksi mutaatiota ei tarvitse pelätä, kunhan sen hoitaa asianmukaisesti.

Jos funktio mutatoi ainoastaan muuttujia, jotka se on itse luonut, ei sillä ole lopputuloksen kannalta toiminnallista merkitystä. Kuitenkin tehokkuusvoitot voivat olla huomattavat versioihin, joissa mutatointia ei harrasteta millään tasolla.

Esimerkkinä funktio, joka ottaa sisään listan avain-arvo pareja, ja luo niistä olion. Näytetään kaksi (funktionaalista) tapaa totettaa funktio.
Versio, joka ei mutatoi muuttujia missään vaiheessa (\ref{code:js_just_mutate_immutable}), ja versio, joka mutatoi vain itse luomiaan muuttujia (\ref{code:js_just_mutate_mutable}).

Tarkkailijan näkökulmasta molemmat funktiot ovat kuitenkin puhtaita.

\begin{code}
    \begin{minted}[highlightlines={8,14}]{javascript}
const entries = Array.from({ length: 1000 }, (_, i) => [
    `key-${i}`,
    `value-${i}`
])

const objectFromEntries = (entries) =>
    entries.reduce((acc, entry) => {
        return { ...acc, [entry[0]]: entry[1] }
    }, {})

console.time('Immutable')
objectFromEntries(entries)
console.timeEnd('Immutable')
// Immutable: 112.718017578125 ms
\end{minted}
    \caption{Funktio, joka ottaa listan avain-arvo -pareja ja luo niistä olion. Olion luonnissa ei käytetä ollenkaan mutatointia.}
    \label{code:js_just_mutate_immutable}
\end{code}

Täysin mutatoimattomalla versiolla kesti luoda tuhannesta avaimesta olio yli 300 kertaa kauemmin, kuin versiolla, joka mutatoi kohtuudella.


\begin{code}
    \begin{minted}[highlightlines={8,15}]{javascript}
const entries = Array.from({ length: 1000 }, (_, i) => [
    `key-${i}`,
    `value-${i}`
])

const objectFromEntries = (entries) =>
    entries.reduce((acc, entry) => {
        acc[entry[0]] = entry[1]
        return acc
    }, {})
          
console.time('Mutable')
objectFromEntries(entries)
console.timeEnd('Mutable')
// Mutable: 0.35693359375 ms
\end{minted}
    \caption{Sama funktio kuin aiempi. Ainoa ero, että pareja iteroidessa luotavaa oliota mutatoidaan.}
    \label{code:js_just_mutate_mutable}
\end{code}

Useimmiten ohjelman tehokkuudella ei ole paljoa merkitystä, ja luotettavuus ja luettavuus ovat edellä. Kuitenkin tässä tilanteessa tehokkuuserot ovat niin merkittävät, että jos 1 000 avain-arvo -parin sijasta, oltaisiin käytetty 100 000, tai 1 vaikka 000 000 avain-arvo -paria, mutatoimaton versio ei välttämättä olisi suoriutunut tehtävästä koskaan. Näitä määriä yritettiin, mutta suoritukset lopetettiin kesken, kun näytti, etteivät tehtävät mene ikinä läpi.


\section{Monadien käyttäminen huomaamatta}


\subsection{Lista on monadi}
.flatmap.flatmap.flatmap.flatmap.flatmap.flatmap

\subsection{Promise on monadi}

.then.then.then.then.then.then.then.then.then.then
\subsection{Maybe-monadi}



\section{Ongelmien mallintaminen}


\section{Virheet mukaan mallintamiseen}

\subsection{Result-monadi}
RESULT MONAD!!!!!!!!!!!!! THIS IS GOLDEN


% Material and Methods
%\clearpage%if the chapter heading starts close to bottom of the page, force a line break and remove the vertical vspace
\vspace{21.5pt}
\chapter{Tulosten analyysi}
TODO

\section{Tutkimustulosten arviointi}
TODO

\section{Kehittäjien kokemukset}
TODO

% Material and Methods
%\clearpage%if the chapter heading starts close to bottom of the page, force a line break and remove the vertical vspace
\vspace{21.5pt}
\chapter{Johtopäätös}

TODO

\section{Pohdinta}

TODO
% fp hyvä, kieliagnostiset datatyypit ja rakenteet hyvä

% Kikkailu huono.

% fp koodin kirjoittaminen vain koska "fp hyvä" on huono

\section{Suositukset}

TODO

\section{Henkilökohtaiset reflektiot}

TODO

% Miten onnistua kirjoittamaan sitä kuuluisaa funktionaalista koodia sitten loppujen lopuksi?

% Varmaan siten, että yrittää miettiä kategoriateoriaa. Kategoriateoria on kovin teoreettista, mutta teoreettinen = hyvä pohja, jos teoria on oikeasti totta eikä harhaunta.

\section{Loppusanat}

%----------------------------------------------------------------------------------------
%	BIBLIOGRAPHY REFERENCES
%----------------------------------------------------------------------------------------

\input{style/biblio.tex}

%----------------------------------------------------------------------------------------
%	APPENDICES
%----------------------------------------------------------------------------------------

% \input{style/appendix.tex}
% %force smaller vertical spacing in table of content
% %!!! There can be some fun depending if the appendices have (sub)sections or not :D
% % You will have to play with these numbers and eventually add the \vspace line  before
% % some \chapter and force another number.
% % To add more fun, time to time the table of content get wrong after a build :(
% \addtocontents{toc}{\vspace{11pt}}
% \pretocmd{\chapter}{\addtocontents{toc}{\protect\vspace{-24pt}}}{}{}

% \liite{1}% This is a hack to have right page numbering for each appendix. Make sure to
% % use a unique number for each appendix.
% \input{sample/Xappendix1.tex}% Sample content to demonstrate appendix in LaTeX. You
% % are safe to delete this lines (and the next samples) once you refreshed your LaTeX
% % skills (and safe to delete the sample folder and all its file too).

% %\addtocontents{toc}{\vspace{11pt}}%fix vertical space for Table of Content
% \liite{2}
% \input{sample/Xappendix2.tex}

% \addtocontents{toc}{\vspace{11pt}}
% \liite{3}
% \input{sample/X_R_example.tex}


%----------------------------------------------------------------------------------------
%	THIS IS THE END
%----------------------------------------------------------------------------------------
\end{document}
